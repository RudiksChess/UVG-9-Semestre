\documentclass[a4paper, 12pt]{article}
\usepackage[utf8]{inputenc} % Kodierung
\usepackage[T1]{fontenc} % Explizite Nennung des Fonts
\usepackage[spanish]{babel} % Sprache
\usepackage{graphicx} % immer benötigt für das Einbinden von Graphiken
\usepackage{blindtext} % Wenn man das Layout prüfen will, kann hier mit \blindtext Text eingfügt werden.
\usepackage{parskip} % Für den Abstand zwischen 2 Absätzen.
\setlength{\parskip}{12pt plus80pt minus10pt} % Genaue Einstellung von parskip
\usepackage{easy-todo} % Mit \todo{} Todos einfügen
\usepackage{csquotes} % Für ordentlichen Anführungszeichen
\usepackage[iso, german]{isodate} % Für eine deutsche Formatierung des Abgabedatums / Eidesstattlicher Erkärung
\usepackage[style=apa, backend=biber]{biblatex} % Biber backend für Literaturverzeichnis
\addbibresource{literatur/bibliography.bib} % Einbinden der Literatur.
\DeclareLanguageMapping{german}{german-apa} % Anpassen Spracheinstellungen im Literaturverzeichnis.
\usepackage[activate={true,nocompatibility},
	final,
	tracking=true,
	kerning=true,
	expansion=true,
	spacing=true,
	factor=1050,
	stretch=25,
	shrink=10]{microtype} % Für die Feineinstellung der Zeichensetzung.
\usepackage{booktabs}
\usepackage{appendix}
\usepackage[rflt]{floatflt}
\usepackage{fancyvrb}
\usepackage[hidelinks]{hyperref} % Klickbare aber nicht markierte Links im PDF
\usepackage{setspace}
\usepackage{fancyhdr} % Für schönere Kopf-/Fußzeilen und Fußnoten.
\usepackage[right=4 cm, left=2.5 cm, top=2.5 cm, bottom=3 cm]{geometry} % Seitenränder
\usepackage{pbox}
\usepackage{tabulary}
\usepackage{amsmath}
\usepackage{amssymb}
\usepackage{amsthm}
\usepackage{hyperref}
\usepackage[skins]{tcolorbox}
\usepackage{xcolor}
\usepackage{svg}
\usepackage{tikz}
\usepackage{mathtools}
\sloppy
\fancyhf{}
\rfoot{\thepage}
\renewcommand{\headrulewidth}{0pt}
%Special cells with linebreaks possible
\newcommand{\specialcell}[2][c]{%
	\begin{tabular}[#1]{@{}t@{}}#2\end{tabular}}
%define blockquote-quotation environment
\renewenvironment{quotation}{
	\leftskip1cm
	\rightskip1cm
	\noindent
	\setstretch{1}
	\small
}

% define Footnote
\renewcommand\footnoterule{\kern-3pt \hrule width 3in height 0.7pt \hskip3pt \kern 2.6pt}
\let\oldfootnote\footnote
\renewcommand\footnote[1]{%
	\oldfootnote{\hspace{2mm}#1}}
%%%%%

% microtyping around some characters

%Extra-spacing around dash, and quotation-marks, and parentheses
\SetExtraKerning[unit=space]
	{
		encoding={*}, family={qhv}, series={b}, size={normalsize,large,Large}
	}
	{
		\textendash={400,400}, % for double-dash
		\textquotedblleft={ ,120}, % for left quotation-mark
		\textquotedblright={170, }, % for right quotation-mark
		"28={ ,250}, % left bracket, add space from right
		"29={300, } % right bracket, add space from left
	}

%%%%%


\makeatletter
\newcommand{\MSonehalfspacing}{%
  \setstretch{1.44}%  default
  \ifcase \@ptsize \relax % 10pt
    \setstretch {1.448}%
  \or % 11pt
    \setstretch {1.399}%
  \or % 12pt
    \setstretch {1.433}%
  \fi
}
\newcommand{\MSdoublespacing}{%
  \setstretch {1.92}%  default
  \ifcase \@ptsize \relax % 10pt
    \setstretch {1.936}%
  \or % 11pt
    \setstretch {1.866}%
  \or % 12pt
    \setstretch {1.902}%
  \fi
}
\newcommand{\MSverbatimspacing}{%
  \setstretch{1.44}%  default
  \ifcase \@ptsize \relax % 10pt
    \setstretch {1}%
  \or % 11pt
    \setstretch {1}%
  \or % 12pt
    \setstretch {1}%
  \fi
}
\makeatother

%---------------------------

\newtcolorbox{cajita}[1][]{
	 #1
}

\newenvironment{sol}
{\renewcommand\qedsymbol{$\square$}\begin{proof}[\textbf{Solución.}]}
	{\end{proof}}

\newenvironment{dem}
{\renewcommand\qedsymbol{$\blacksquare$}\begin{proof}[\textbf{Demostración.}]}
	{\end{proof}}

\newtheorem{problema}{Problema}
\newtheorem{definicion}{Definición}
\newtheorem{ejemplo}{Ejemplo}
\newtheorem{teorema}{Teorema}
\newtheorem{corolario}{Corolario}[teorema]
\newtheorem{lema}[teorema]{Lema}
\newtheorem{prop}{Proposición}
\newtheorem*{nota}{\textbf{NOTA}}
\renewcommand\qedsymbol{$\blacksquare$}

\begin{document}
\newgeometry{margin=2.5cm}
\begin{titlepage}
\thispagestyle{empty}
\newcommand{\HRule}{\rule{\linewidth}{0.5mm}}
\hspace{1cm}
\center

\textsc{\huge Universidad del Valle de Guatemala}\\[2.0cm]
\textsc{\Large 1 semestre - 2023}\\[0.8cm]
\MSonehalfspacing
\textsc{Licenciatura en matemática aplicada}\\[1.0cm]

\HRule\\[1.4cm]
\MSdoublespacing
{ \huge \bfseries Topología}\\[0.2cm]
{ \large Catedrático: Dorval Carías}\\[0.3cm] % Falls nicht benötigt, einfach auskommentieren
\HRule \\[2.4cm]
\MSonehalfspacing

\begin{minipage}[t]{0.8\textwidth}
	\begin{itemize}
	\item[\emph{Estudiante:}] Rudik Roberto Rompich Cotzojay
	\item[\emph{Carné:}] 19857
	\item[\emph{Correo:}] rom19857@uvg.edu.gt
	\end{itemize}
\end{minipage}

\vspace{2.9cm}

\flushright \today
\end{titlepage}
\restoregeometry

\setstretch{3}
\microtypesetup{protrusion=false}
\tableofcontents
\microtypesetup{protrusion=true}
\thispagestyle{empty}

\MSonehalfspacing
\newpage
\setcounter{page}{1}
\pagestyle{fancy}
\setcounter{page}{1}
%----------------------------------------


\subsection{Nets}

\begin{prop}
    Sea $D$ un conjunto $y\leq$ una relación definida sobre $D$ que satisface: 
    \begin{enumerate}
        \item $\leq$ es reflexiva, $x\leq x,\forall x\in D$. 
        \item $\leq$ es transitiva, si $x\leq y$ y $y\leq z\implies x\leq z$. 
        \item $\leq$ es dirigida, si $x,y\in D\implies \exists z\in D\ni x\leq z$ y $y\leq z$.  
    \end{enumerate}
    Entonces el para $(D,\leq)$ es un conjunto dirigido. 
\end{prop}



\begin{definicion}
    Una red en un conjunto $X$ es un mapeo 
    $$w:D\to X$$
    donde $(D,\leq)$ es un conjunto dirigido.  
\end{definicion}

\begin{nota}
    Cada sucesión sobre $X$ es una red. 
\end{nota}

\begin{definicion}[Convergencia]
    Si $(X,\tau)$ es un espacio topológico y $w:D\to X$ es una red, se dice que $w$ converge a $x\in X$, si para cada abierto $U$ que contiene a $x$, existe $d\in D\ni T_\alpha=\{w(e):d\leq e\in D\}\subseteq U$. 
\end{definicion}

\begin{nota}
    $w\to x$ (la red converge a $x$), o bien $x$ es punto límite de $w$. 
\end{nota}

\begin{teorema}
    Sea $(X,\tau)$ un espacio topológico y sea $A\subseteq X$, entonces $x\in \mathbb{A}$ ssi existe una red $w$ sea $A\ni w\to x$.
\end{teorema}


\begin{nota}
    Un subconjunto $D'$ de un subconjunto dirigido $D$ es cofinal, si $\forall d\in D\exists e\in D'\ni d\leq e$. 
\end{nota}

\begin{definicion}
    Sean $w:D\to X$ y $v: E\to x$ redes sobre $X$ (donde $(D,\leq)$ y $(E,\leq )$ son conjuntos dirigidos). Se dice que $v$ es una subred de $w$ si existe una función $h:E\to D\ni$ 
    \begin{enumerate}
        \item $h$ es monótona, es decir, $\alpha\leq \beta\implies h(\alpha)\leq h(\beta)$
        \item $h$ es cofinal (es decir, $h(E)$ es cofinal con $D$).
        \item $v(\alpha)=w(h(\alpha)),\forall \alpha\in E$.
    \end{enumerate}
    
\end{definicion}

\begin{definicion}[Subsucesión]
    Una subsucesión de $(X_n)$ es una sucesión de la forma $(X_{n_k})$, es decir, dadad $(X_n)$, la subsucesión es de la forma $(X_{h_k})$, donde $h$ es una función creciente, $h:\mathbb{Z}^+\to \mathbb{Z}^+$, y donde $h$ no es acotada (i.e. su rango es cofinal con $\mathbb{Z}^+$)
\end{definicion}
\begin{definicion}
    Sea $X$ un conjunto. Una colección $\mathcal{F}\subseteq P(X)$ es un filtro sobre $X$ si se satisfacen: 
    \begin{enumerate}
        \item $\varnothing\not\in \mathcal{F}$
        \item Si $A\in \mathcal{F}$ y $A\subseteq B\implies B\in \mathcal{F}$ 
        \item Si $A,B\in \mathcal{F}\implies A\cap B\in \mathcal{F}$
    \end{enumerate}
\end{definicion}

\begin{ejemplo}
    Sea $\mathcal{F}=\{x\}$, siempre que $X\neq \varnothing$ es filtro trivial. 
\end{ejemplo}

\begin{ejemplo}
    Sea $X\neq \varnothing$, $x\in X$. Entonces $X= \{A\ni A\subseteq X\wedge x\in A\}$
\end{ejemplo}
\begin{ejemplo}
    Sean $(X,\tau)$ un espacio topológico, $x\in X$ 
    $$\mathcal{F}_x=\{A\subseteq X: \exists U\in \tau \ni x\in U\subseteq A\}$$
    es el filtro de vecindades de $X$. 
\end{ejemplo}

\begin{ejemplo}
    Sea $X$ un conjuhnto infinito y sea 
    $$\mathcal{F}=\{A\subseteq X\ni X-A \text{ es finito}\}$$
    $\mathcal{F}$ se le conoce como el filtro de Frechet. 
\end{ejemplo}

\begin{cajita}
    Dado un conjunto $X$, cualquier colección $S\subseteq P(X)$ tiene la propiedad de intersección finita (PIF) si para todos  $A_1,A_2,\cdots, A_n\in S$, se tiene que $\bigcap_{k=1}^n A_k\neq \varnothing$. 

\end{cajita}
\begin{nota}
    Cualquier coleccion $S\subseteq P(X)$ con la PIF genera un filtro que la contiene. 
\end{nota}

\begin{nota}
    Sea $F(X)$ la colección de todos los filtros sobre $X$. Sea $\leq$ la relación de contención, entonces $(F(X),\leq)$ es un conjunto parcialmente ordenado. Este orden no puede ser lineal ($x\not\subset y$ y $y\not\subset x$)
\end{nota}

%------------------------- clase 

\begin{ejemplo}
    $M\subseteq X\implies \mathcal{F}_n=\{A\subseteq X\ni M\subseteq A\}$ es el filtro principal generado por $M$. 
\end{ejemplo}
\begin{ejemplo}
    $M=\{x\},x\in X\implies X=\mathcal{F}_x=\{A\subseteq X\ni x\in A\}$. 
\end{ejemplo}

%----------------- clase 

\begin{teorema}
    Sea $X\neq \varnothing$ y $\mathcal{F}_\alpha\in F(X),\alpha\in I$. Entonces $\bigcap_\alpha \mathcal{F}_\alpha \in F(X)$

\end{teorema}

\begin{nota}
    Sea $X\neq\varnothing$ y sea $A\not\subseteq X\implies$ Considere $B=X-A=A^c$ y a los filtros $\mathcal{F}_A$ y $\mathcal{F}_B$. Entonces, $\mathcal{F}_A\cup \mathcal{F}_B$ no es filtro. En efecto, como $A\subset \mathcal{F}_A$ y $B\subset \mathcal{F}_B\implies A\cap B=\varnothing\implies \varnothing \mathcal{F}_A\cup \mathcal{F}_B$. 
\end{nota}

\begin{teorema}
    Sea $X$ un conjunto y $U(x)$ una colección de filtros sobre $X$. Si para cualesquiera $\mathcal{F}_1,\mathcal{F}_2\in U(x)$ se tiene que $\mathcal{F}_1\subset \mathcal{F}_2$ o $\mathcal{F}_2\subset \mathcal{F}_1\implies \bigcup U(x)$ es filtro. 
\end{teorema}

\begin{prop}
    Sea $X$ un conjunto y $\mathcal{F}, \mathcal{G}$ filtros sobre $X$. Entonces, $\mathcal{F}\bigcup_* G:=\{F\cup G:F\in \mathcal{F}\wedge G\in \mathcal{G}\}$ es filtro sobre $X$. 
\end{prop}


\begin{cajita}
    \begin{nota}[Escolio]
        Sea $\mathcal{F}$ un filtro sobre $X\neq\varnothing$. Entonces,  $\mathcal{F}\cup \{\varnothing\}$ es una topología sobre $X$. 
    \end{nota}    
\end{cajita}

\begin{definicion}[Ultrafiltros]
    Sea 
    \begin{itemize}
        \item Un filtro sobre $X$ es un ultrafiltro  si se cumple $\forall A\subset X $, se tiene que $A\in \mathcal{F}$ o $A^c\in \mathcal{F}$.
        \item Un ultrafiltro es un filtro maximal en $(F(X), \subseteq)$; i.e. un ultrafiltro es un filtro $U$ sobre $X$ tal que si $\mathcal{G}$ es un filtro sobre $X$ tal que $X\ni U\subseteq G\implies U=G$ 
    \end{itemize}
\end{definicion}

\begin{definicion}
    $\mathcal{F}$ converge a $x$ ($\mathcal{F}\to x$) si $\forall V\in N(x)\exists F\in \mathcal{F}\ni F\subset V$. 
\end{definicion}

\begin{definicion}
    Sean $(X,\leq)$ un conjunto parcialmente ordenado, $a\in X$ y $A\subset X$. Se dice que $a$ es un elemento maximal de $A$, si $a\in A$ y si $a\leq b$, para todo $b\in A\implies a=b$.  
\end{definicion}

\begin{definicion}
    Sean $(X,\leq)$ un conjunto parcialmente ordenado y $C\subset X$. Se dice que $C$ es cadena en $X$, si $\forall a,b\in C$ se cumple $a\leq b$ o $b\leq a$.  
\end{definicion}

\begin{lema}[De Zorn]
    Si $X$ es un conjunto vacío y parcialmente ordenado $\ni$ cada cadena en $X$ tiene cota superior, entonces $X$ tiene un elemento maximal. 
\end{lema}

\begin{definicion}
    Sea $X\neq \varnothing$. Una familia $U\subseteq P(X)$ (Potencia: $2^X$) es un ultrafiltro si se cumplen: 
    \begin{enumerate}
        \item $\mathcal{U}$ es filtro. 
        \item Si $\mathcal{F}$ es un filtro sobre $X$ ral que $\mathcal{U}\subseteq \mathcal{F}\implies \mathcal{U}= \mathcal{F}$.  $\mathcal{U}$ es un filtro maximal
    \end{enumerate} 
\end{definicion}
\begin{ejemplo}
    Sea $X$ un espacio topológico. Si $x\in X$, entonces $\mathcal{F}_x = \{A\subseteq X\ni x\in A\}$ es un ultrafiltro de $X$. En efecto: 
    \begin{enumerate}
        \item $\mathcal{F}_x$ es un filtro. 
        \item Sea $\mathcal{G}$ un filtro sobre $X\ni \mathcal{F}_x\subseteq \mathcal{G}$. Como $\{x\}\in \mathcal{F}$ y $\mathcal{F}_x\subset \mathcal{G}\implies $ para cada $G\in \mathcal{G}\implies G\cap\{x\}\neq \varnothing\implies x\in G,\forall G\in \mathcal{G}\implies G\in \mathcal{F}_x\implies \mathcal{G}\subset \mathcal{F}_x\implies \mathcal{F}$ es un ultrafiltro sobre $X$.
        \item  Sea $X$ un conjunto vacío y $A\subset X$, con al menos dos puntos. $\implies \mathcal{F}_A=\{F\subset X\ni A\subset F\}$ no es un ultra filtro. Entonces $M\in \mathcal{F}_A\implies A\subset M\implies x\in M \implies M\in \mathcal{F}_x\implies \mathcal{F}_A\subset \mathcal{F}_x$. Además, como $\{x\}\in \mathcal{F}_x$ y además, $\{x\}\not\in \mathcal{F}_A\implies \mathcal{F}_x\not\subset \mathcal{F}_A\implies \mathcal{F}_A$ no puede ser ultrafiltro en $X$. 
    \end{enumerate}
\end{ejemplo}

\begin{teorema} (Tarski, 1930)
    Sea $X$ un conjunto y $\mathcal{F}$ un filtro sobre $X$. Entonces, existe un ultrafiltro $\mathcal{U}$ sobre $X$ tal que $\mathcal{F}\subset \mathcal{U}$. 

    
\end{teorema}



\begin{definicion}
    Una subcolección $\beta\subset \mathcal{F}$ es una base del filtro $\mathcal{F}$ si ocurre $\forall F\in \mathcal{F}\exists B\in \beta\ni B\subset F$.
\end{definicion}

\begin{teorema}
    Una familia $\beta$ de subconjuntos no vacíos de $X$ es base de algún filtro sobre $X$ ssi $\forall B_1,B_2\in \beta \ni B_3 \subset \beta \ni B_3 = B_1\cap B_2$.
\end{teorema}

 %--------------------------------


 %cosas

 \begin{teorema}
    Sea $X,Y$ espacios topológicos, $\mathcal{F}$ un filtro sobre $X$ y un mapeo $f:X\to Y$. Entonces 
    $$\beta_{\mathcal{F}}=\{f(F):F\in \mathcal{F}\}$$
    es una base para filtros en $Y$. 
 \end{teorema}

 \begin{teorema}
    Sean $X,Y$ espacios topológicos, $\mathcal{F}$ un filtro sobre $Y$ y un mapeo $f:X\to Y$. Si, $\forall F\in \mathcal{F}$ se tiene que $f^{-1}(F)\neq 0$, entonces $\beta=\{f^{-1}(F):F\in \mathcal{F}\}$ es una base de filtros para $X$. 

 \end{teorema}

 \begin{teorema}
    Sea $X\neq \varnothing, \mathcal{F}$ un filtro sobre $X$ y $E\subset X$. Si $\beta=\{F\cap E: F\in \mathcal{F}\}$ y $\beta' =\{F\cap (X-E):F\in \mathcal{F}\}$, entonces: 
    \begin{enumerate}
        \item Si $F\cap E\neq \varnothing,\forall F\in \mathcal{F}\implies \beta$ es una base de filtros sobre $X$. 

        \item Si $\exists F\in \mathcal{F}\ni F\cap E\neq \varnothing \implies \beta '$ es base de filtros sobre $X$. 

    \end{enumerate}
 \end{teorema}

 \begin{definicion}
    Sea $(X,\tau)$ un espacio topológico $F$ un filtro sobre $X$, $x\in X$. Entonces: 
    \begin{enumerate}
        \item Se dice que $\mathcal{F}$ converge a $x$, $\mathcal{F}\to x$, si $\forall V\in N(x)\exists F\in \mathcal{F}\ni F\subset V$
        \item Se dice que $x$ es un punto de acumulación de $\mathcal{F}$, si $\forall F\in \mathcal{F}$ y $\forall V \in N(x)$, se cumple que $F\cap V\neq \varnothing$. Notación: $F\geq x$
    \end{enumerate}
 \end{definicion}


 \begin{teorema}
    Sean $X$ un conjunto y $U$ un filtro sobre $X$. Entonces, los enunciados siguientes son equivalentes: 
    \begin{enumerate}
        \item $\mathcal{U}$ es ultrafiltro
        \item Para cada $E\subset X\ni E\cap F\neq \varnothing,\forall F\in \mathcal{U}$, se tiene que $E\in \mathcal{U}$.
        \item Si $E\subset X\implies E\in \mathcal{U}$ o $ X-E\in \mathcal{U}$
        \item Si $A,B\subset X$ y $A\cup B \in \mathcal{U}\implies A\in \mathcal{U}$ o $B\in \mathcal{U}$. 
    \end{enumerate}
\end{teorema}

\begin{definicion}
    Sean $(X,\tau)$ un espacio topológico y $\mathcal{F}$ un filtro sobre $X$ y $x\in X$.Entonces, $\mathcal{F}\to x$
    , si $\forall V\in N(x)\exists F\in \mathcal{F}\ni F\subset V$.
\end{definicion}

\begin{teorema}
    Sea $(X,\tau)$ un espacio topológico $\mathcal{F}$ un filtro sobre $X$ y $x\in X$. Entonces, $\mathcal{F}\to x\iff N(x)\subset F$. 
\end{teorema}

\begin{ejemplo}
    Sean $X$ un espacio topológico y $x\in X\implies \mathcal{F}(x)=N(x)$ converge a $x$. 
\end{ejemplo}

\begin{ejemplo}
    Sea $X$ un espacio indiscreto. Como $X$ es la única vecindad disponible para convergencia, entonces cualquier filtro sobre $X$ converge a cada punto de $X$. 
\end{ejemplo}

\begin{ejemplo}
    Considere el espacio de Sierpinski. es decir: $X=\{0,1\}$ y $\tau=\{X,\varnothing, \{0\}\}$ y sea $\mathcal{F}=N(x)$. 
    \begin{itemize}
        \item Sea $x=0\implies N(x)\to 0$; $N(X)\to 1$
    \end{itemize}
\end{ejemplo}

\begin{teorema}
    Sea $X$ un espacio topológico $x\in X$ y $\mathcal{F}$ un filtro sobre $X\ni \mathcal{F}\to x$. Si $G$ es un filtro sobre $X\ni F\subset G\implies G\to x$. 
\end{teorema}

\begin{definicion}
    Se dice que $x$ es un punto de acumulación del filtro $\mathcal{F}$ sobre $X$, si $\forall F\in \mathcal{F}$ y $\forall V\in N(x)$, se cumple que $F\cap V\neq \varnothing$ 
    \begin{nota}
        $F>x$
    \end{nota}
\end{definicion}


%----------------------------------------
\newpage

\end{document}