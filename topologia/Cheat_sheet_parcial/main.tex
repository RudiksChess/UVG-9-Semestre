\documentclass[a4paper, 12pt]{article}
\usepackage[utf8]{inputenc} % Kodierung
\usepackage[T1]{fontenc} % Explizite Nennung des Fonts
\usepackage[spanish]{babel} % Sprache
\usepackage{graphicx} % immer benötigt für das Einbinden von Graphiken
\usepackage{blindtext} % Wenn man das Layout prüfen will, kann hier mit \blindtext Text eingfügt werden.
\usepackage{parskip} % Für den Abstand zwischen 2 Absätzen.
\setlength{\parskip}{12pt plus80pt minus10pt} % Genaue Einstellung von parskip
\usepackage{easy-todo} % Mit \todo{} Todos einfügen
\usepackage{csquotes} % Für ordentlichen Anführungszeichen
\usepackage[iso, german]{isodate} % Für eine deutsche Formatierung des Abgabedatums / Eidesstattlicher Erkärung
\usepackage[style=apa, backend=biber]{biblatex} % Biber backend für Literaturverzeichnis
\addbibresource{literatur/bibliography.bib} % Einbinden der Literatur.
\DeclareLanguageMapping{german}{german-apa} % Anpassen Spracheinstellungen im Literaturverzeichnis.
\usepackage[activate={true,nocompatibility},
	final,
	tracking=true,
	kerning=true,
	expansion=true,
	spacing=true,
	factor=1050,
	stretch=25,
	shrink=10]{microtype} % Für die Feineinstellung der Zeichensetzung.
\usepackage{booktabs}
\usepackage{appendix}
\usepackage[rflt]{floatflt}
\usepackage{fancyvrb}
\usepackage[hidelinks]{hyperref} % Klickbare aber nicht markierte Links im PDF
\usepackage{setspace}
\usepackage{fancyhdr} % Für schönere Kopf-/Fußzeilen und Fußnoten.
\usepackage[right=4 cm, left=2.5 cm, top=2.5 cm, bottom=3 cm]{geometry} % Seitenränder
\usepackage{pbox}
\usepackage{tabulary}
\usepackage{amsmath}
\usepackage{amssymb}
\usepackage{amsthm}
\usepackage{hyperref}
\usepackage[skins]{tcolorbox}
\usepackage{xcolor}
\usepackage{svg}
\usepackage{tikz}
\usepackage{mathtools}
\sloppy
\fancyhf{}
\rfoot{\thepage}
\renewcommand{\headrulewidth}{0pt}
%Special cells with linebreaks possible
\newcommand{\specialcell}[2][c]{%
	\begin{tabular}[#1]{@{}t@{}}#2\end{tabular}}
%define blockquote-quotation environment
\renewenvironment{quotation}{
	\leftskip1cm
	\rightskip1cm
	\noindent
	\setstretch{1}
	\small
}

% define Footnote
\renewcommand\footnoterule{\kern-3pt \hrule width 3in height 0.7pt \hskip3pt \kern 2.6pt}
\let\oldfootnote\footnote
\renewcommand\footnote[1]{%
	\oldfootnote{\hspace{2mm}#1}}
%%%%%

% microtyping around some characters

%Extra-spacing around dash, and quotation-marks, and parentheses
\SetExtraKerning[unit=space]
	{
		encoding={*}, family={qhv}, series={b}, size={normalsize,large,Large}
	}
	{
		\textendash={400,400}, % for double-dash
		\textquotedblleft={ ,120}, % for left quotation-mark
		\textquotedblright={170, }, % for right quotation-mark
		"28={ ,250}, % left bracket, add space from right
		"29={300, } % right bracket, add space from left
	}

%%%%%


\makeatletter
\newcommand{\MSonehalfspacing}{%
  \setstretch{1.44}%  default
  \ifcase \@ptsize \relax % 10pt
    \setstretch {1.448}%
  \or % 11pt
    \setstretch {1.399}%
  \or % 12pt
    \setstretch {1.433}%
  \fi
}
\newcommand{\MSdoublespacing}{%
  \setstretch {1.92}%  default
  \ifcase \@ptsize \relax % 10pt
    \setstretch {1.936}%
  \or % 11pt
    \setstretch {1.866}%
  \or % 12pt
    \setstretch {1.902}%
  \fi
}
\newcommand{\MSverbatimspacing}{%
  \setstretch{1.44}%  default
  \ifcase \@ptsize \relax % 10pt
    \setstretch {1}%
  \or % 11pt
    \setstretch {1}%
  \or % 12pt
    \setstretch {1}%
  \fi
}
\makeatother

%---------------------------

\newtcolorbox{cajita}[1][]{
	 #1
}

\newenvironment{sol}
{\renewcommand\qedsymbol{$\square$}\begin{proof}[\textbf{Solución.}]}
	{\end{proof}}

\newenvironment{dem}
{\renewcommand\qedsymbol{$\blacksquare$}\begin{proof}[\textbf{Demostración.}]}
	{\end{proof}}

\newtheorem{problema}{Problema}
\newtheorem{definicion}{Definición}
\newtheorem{ejemplo}{Ejemplo}
\newtheorem{teorema}{Teorema}
\newtheorem{corolario}{Corolario}[teorema]
\newtheorem{lema}[teorema]{Lema}
\newtheorem{prop}{Proposición}
\newtheorem*{nota}{\textbf{NOTA}}
\renewcommand\qedsymbol{$\blacksquare$}

\begin{document}
\newgeometry{margin=2.5cm}
\begin{titlepage}
\thispagestyle{empty}
\newcommand{\HRule}{\rule{\linewidth}{0.5mm}}
\hspace{1cm}
\center

\textsc{\huge Universidad del Valle de Guatemala}\\[2.0cm]
\textsc{\Large 1 semestre - 2023}\\[0.8cm]
\MSonehalfspacing
\textsc{Licenciatura en matemática aplicada}\\[1.0cm]

\HRule\\[1.4cm]
\MSdoublespacing
{ \huge \bfseries Topología}\\[0.2cm]
{ \large Catedrático: Dorval Carías}\\[0.3cm] % Falls nicht benötigt, einfach auskommentieren
\HRule \\[2.4cm]
\MSonehalfspacing

\begin{minipage}[t]{0.8\textwidth}
	\begin{itemize}
	\item[\emph{Estudiante:}] Rudik Roberto Rompich Cotzojay
	\item[\emph{Carné:}] 19857
	\item[\emph{Correo:}] rom19857@uvg.edu.gt
	\end{itemize}
\end{minipage}

\vspace{2.9cm}

\flushright \today
\end{titlepage}
\restoregeometry

\setstretch{3}
\microtypesetup{protrusion=false}
\tableofcontents
\microtypesetup{protrusion=true}
\thispagestyle{empty}

\MSonehalfspacing
\newpage
\setcounter{page}{1}
\pagestyle{fancy}
\setcounter{page}{1}
%----------------------------------------


\section{Topología}

\begin{definicion}
    Sea $X\neq \varnothing$. Una clase $\tau$ de subconjunto de $X$ es una topología sobre $X$, se cumple: 
    \begin{enumerate}
        \item $\varnothing,X\in \tau$
        \item La unión de una clase arbitraria de conjuntos en $\tau$ es un miembro de $\tau$. 
        \item La intersección de una clase finita de miembros de $\tau$ está en $\tau$. 
    \end{enumerate}
    Los miembros de $\tau$ son los abiertos de $X$. 
\end{definicion}

\begin{cajita}
    \begin{enumerate}
        \item El par $(X,\tau)\footnotetext{estructura topológica}$ es un espacio topológico. 
        \item A los elementos de $X$ se les llama puntos. 
    \end{enumerate}
\end{cajita}

\begin{ejemplo}
    \begin{enumerate}
        \item Sea $X\neq \varnothing\implies \tau=P(X)$ es una topología sobre $X$. A $\tau$ se le llama topología discreta de $X$, y $(X,\tau)$ es un espacio discreto. 

        \item Sea $X\neq \varnothing\implies \tau =\{\varnothing,X\}$ es una topología sobre $X$. A $\tau$ se le llama topología indiscreta, y $(X,\tau)$ es un espacio indiscreto. 
        \item $X=\mathbb{R}^2$ y $\tau$ es la colección de abiertos de $\mathbb{R}^2$ definido en términos de la métrica usual. A $\tau$ se le llama topología usual de $\mathbb{R}^2$. 
        \item Sea $X=\{a,b,c,d,e\}$.
        \begin{enumerate}
            \item Sea $\tau_1=\{X,\varnothing,\{a\},\{c,d\},\{a,c,d\},\{b,c,d,e\}\}\implies \tau_1$ es una topología sobre $X$.
            \item Sea $\tau_2=\{X,\varnothing,\{a\},\{c,d\},\{a,c,d\},\{b,c,d\}\}$. Note que $\{a\}\cup \{b,c,d\}=\{a,b,c,d\}\not\in \tau_2\implies \tau_2$ no es topología sobre $X$
            \item Sea $X$ un conjunto infinito y sea $\tau$ el vacío junto con la colección de subconjunto de $X$ cuyos complementos son finitos. $\tau$ es una topología sobre $X$, y se llama topología cofinita sobre $X$. 
        \end{enumerate}  
    \end{enumerate}
\end{ejemplo}

\begin{cajita}
    \begin{nota}
        Un espacio metrizable es un espacio topológico $X$ con la propiedad que existe una métrica que genera los abiertos de la topología dada. 
    \end{nota}
    \begin{problema}
        ¿Qué tipos de espacios topológicos son metrizables?
    \end{problema}
\end{cajita}

\begin{prop}
    Si $\tau_1$ y $\tau_2$ son topologías sobre $X$, entonces $\tau_1\cap \tau_2$ es topología sobre $X$.
    \begin{dem}
        \begin{enumerate}
            \item Como $\tau_1$ y $\tau_2$ son topologías, etnonces: $X,\varnothing\in \tau_1$ y $X,\varnothing\in \tau_2\implies X\in \tau_1\cap \tau_2$ y $\varnothing\in \tau_1\cap \tau_2$. 
            \item Sea $\{G_i\}_{i\in I}$ una subcolección de $\tau_1\cap \tau_2\implies G_i\in \tau_1,\forall i\in I \implies \bigcup_i G_i\in \tau_i$ y $G_i\in \tau_2,\forall i\in I\implies\bigcup_i G_i\in \tau_2$. Entonces $\bigcup_i G_i\in \tau_1\cap \tau_2$
            \item Sea $G_1$ y $G_2\in \tau_1\cap \tau_2\implies G_1\in\tau_1$ y $G_1\in \tau_2$. $G_2\in \tau_1$ y $G_2\in \tau_2$. Entonces $G_1\cap G_2\in \tau_1$ y $G_1\cap G_2\in \tau_2\implies G_1\cap G_2\in \tau_1\cap\tau_2$. Entonces, $\tau_1\cap \tau_2$ es una topología sobre $X$. 
        \end{enumerate}
        
    \end{dem}
\end{prop}

\begin{nota}
    Sea $X=\{a,b,c\}$ y sean:
    \begin{itemize}
        \item $\tau_1=\{X,\varnothing,\{a\}\}$, $\tau_2=\{X,\varnothing, \{b\}\}$. Entonces, $\tau_1\cup \tau_2=\{X,\varnothing,\{a\},\{b\}\}$, pero $\{a,b\}\not\in \tau_1\cup\tau_2$. $\therefore\tau_1\cup\tau_2$ no es topología sobre $X$.
    \end{itemize}
\end{nota}

\begin{ejemplo}
    Sea $f:X\to Y$, donde $X$ es un conjunto no vacío y $Y$ es el espacio topológico de $(Y,\tau')$. Entonces $\tau=\{f^{-1}(G):G\in \tau'\}$ es una topología sobre $X$. En efecto:
    \begin{enumerate}
        \item $\varnothing\implies \tau'\implies f^{-1}(\varnothing)=\varnothing\in\tau$. $Y\in \tau'\implies f^{-1}(Y)=X\in\tau$
        \item Sea $\{G_i\}$ una subclase de $\tau$. Como $G_i\in \tau,\forall \implies \exists H_i\in \tau'\ni G_i=f^{-1}(H_i)\implies \cup_i G_i=\cup_{i}f^{-1}(H_i)=f^{-1}(\underbrace{\bigcup_i H_i}_{\in \tau'})\in \tau $
    \end{enumerate}
\end{ejemplo}

\begin{definicion}
    Sean $X$ y $Y$ espacios topológicos y $f$ un mapeo de $X$ en $Y$. Se dice que $f$ es continua si $f^{-1}(G)$ es un abierto de $X$ para cada abierto de $G$ de $Y$. 
\end{definicion}

\begin{definicion}
    Se dice que el mapeo es abierto si, para cada abierto $G$ de $X$, se cumple que $f(G)$ es abierto de $Y$. 
\end{definicion}

\begin{definicion}
    Si $f$ es continuo, entonces $f(x)$ es la imagen continua de $X$ bajo $f$. 
\end{definicion}

\begin{definicion}[Homeomorfismo]
    Un homeomorfismo es un mapeo biyectivo y bicontinuo (continuo y abierto) entre espacios topológicos. En este caso, los espacios son homeomorfos. 
\end{definicion}

\begin{cajita}
    \begin{nota}
        Una propiedad topológica es una propiedad que si la tiene el espacio topológico $X$, la tiene  también cualquier espacio homeomorfo a $X$
    \end{nota}
\end{cajita}


%----

\begin{nota}
    Sea $A$ un subconjunto no vacío del espacio topológico $(X,\tau)$. Considerese la clase: 
    $$\tau_A=\{A\cap G: G\in\tau \text{es abierto de $X$}\}$$
    Entonces, $\tau_A$ es una topologia sobre $A$, la cual se llama topologia relativa sobre $A$.
\end{nota}

\begin{definicion}
    El par $(A,\tau_A)$ es un espacio topologico y se dice es un subespacio de $X$, 
    \begin{enumerate}
        \item $\varnothing\in \tau\implies A\cap \varnothing=\varnothing\in\tau_A$ y $X\in \tau\implies A\cap X=A\in \tau_A$.
        \item Sea $\{G_i\}_{i\in I}$ una colección de miembros de $\tau_A\implies \exists H_i\in \tau\ni G_i=A\cap H_i, \forall i\implies \bigcup_i G_i=\bigcup_i(A\cap H_i)= A\cap\left(\underbrace{\bigcup_{i}H_i}_{\in\tau}\right)\in\tau_A$
        \item Sean $G_1,G_2\in \tau_A\implies\exists H_i\in\tau\ni G_i=A\cap H_i$, $i=1,2$. Entonces, $G_1\cap G_2=(A\cap H_1)\cap (A\cap H_2)=A\cap (\underbrace{H_1\cap H_2}_{\in\tau})\in\tau_A\implies \tau_A$ es topología sobre $A$. 
    \end{enumerate}
\end{definicion}

\begin{ejemplo}
    Tenemos, 
    \begin{enumerate}
        \item Sea $\tau$ la topología usual de $\mathbb{R}$ y considere la topología relativa $\tau_{\mathbb{Z}^+}$ (en este caso, $\mathbb{Z}^+\subset \mathbb{R})$. Nótese que $\{n
        _0\}$ es abierto, la unión de unitarios es abierto de $\tau_{\mathbb{Z}^+}\implies \tau_{\mathbb{Z}^+}$ es la topología discreta de $\mathbb{Z}^+$.
        \item Considere $(\mathbb{R},\tau)$, donde $\tau$ es la topología usual de $\mathbb{R}$ y  sea $I=[0,1]$. Entonces, 
        \begin{enumerate}
            \item $(1/2,1]=[0,1]\cap (1/2,2)\in \tau_I$
            \item $(1/2,2/3)=[0,1]\cap (1/2,2/3)\in \tau_I$
            \item $(0,1/2]\not\in \tau_I$, ya que no existe un abierto $G\in \tau\ni (0,1/2]=I\cap G$.
        \end{enumerate}
        \item Sea $X=\{a,b,c,d,e\}$ y sea 
        $$\tau=\{X,\varnothing,\{a\},\{a,b\},\{a,b,c,d\},\{a,b,e\}\}$$
        \begin{cajita}
            $\tau_A=\{A,\varnothing,\{a\},\{a,c\},\{a,e\}\}$
        \end{cajita}
        Considere $A=\{a,c,e\}$ entonces: \begin{itemize}
            \item $A\cap X=A$
            \item $A\cap\varnothing=\varnothing$
            \item $A\cap\{a\}=\{a\}$
            \item $A\cap\{a,b\}=\{a\}$
            \item $A\cap\{a,c,d\}=\{a,c\}$
            \item $A\cap \{a,b,c,d\}=\{a,c\}$
            \item $A\cap\{a,b,e\}=\{a,e\}$
        \end{itemize}
    \end{enumerate}
\end{ejemplo}

%---------
\subsubsection{Objeto de estudio de la topología}: Estudio de todas las propiedades topológicas de los espacios

\begin{definicion}
    Sea $(X,\tau)$ un espacio topológico. Un subconjunto $A\subset X$ es cerrado si y solo si $A^c\in\tau $.
\end{definicion}

\begin{ejemplo}
    Sea $(X,\tau)$ un espacio discreto. Sea $A\subset X\implies A\in\tau\implies A^c\subset X\implies A^c\in \tau\implies A$ es cerrado. Entonces, $A\subset X$ es abierto y cerrado en $X$. 
\end{ejemplo}

\begin{cajita}
    \begin{nota}
        Sea $(X,\tau)$ un espacio topológico,
        \begin{enumerate}
            \item $\phi\in \tau\implies \phi^c=X$ es cerrado. $X\in\tau\implies X^c=\phi$ es cerrado.
            \item Considere una familia arbitraria $\{F_i\}$ de cerrados en $\tau\implies \{F_i^c\}\subset \tau\implies \bigcup_i F_i^c\in \tau \implies \left(\bigcup_i F_i^c\right)^c=\bigcap_i F_i$ es cerrado. 
            \item Sean $F_1$ y $F_2$ cerrados en $\tau\implies F_1^c$ y $F_2^c\in\tau\implies F_1^c\cap F_2^c\in \tau\implies (F_1^c\cap F_2^c)^c=F_1\cup F_2$ es cerrado.   
        \end{enumerate}
    \end{nota}
\end{cajita}


\begin{definicion}
    Sea $X$ un espacio topológico: 
    \begin{enumerate}
        \item Una vecindad de un punto (o de un conjunto), es un abierto de $X$ que contiene al punto (o al conjunto).
        \item Sea $A\subseteq X$. Un punto $x$ en $A$ es aislado si existe una vecindad de $x$ que no contiene ningún otro punto de $A$. 
        \item Sea $A\subseteq X$. Un punto de $y\in X$ es un punto límite de $A$ si, $\forall G\in \tau\ni y\in G$, se tiene que $(G-\{y\})\cap A\neq\varnothing$. 
        \begin{cajita}
            EL conjunto de puntos límite de $A$ se llama derivado de $A$, $(A',D(A))$.  
        \end{cajita}
        \item Sea $A\subseteq X$. La cerradura de $A$, denotado $\overline{A}$, es el cerrado más pequeño que contiene a $A$. Es decir, si $F_i$ son los cerrados de $X$ que contiene a $A\implies \overline{A}=\bigcap_i F_i$. 
        \begin{cajita}
            Tenemos:
            \begin{enumerate}
                \item $A\subseteq \overline{A}$
                \item Si $A$ es cerrado $\implies A=\overline{A}$.
            \end{enumerate}
        \end{cajita}
        \item Un subconjunto $A$ de $X$ es denso (siempre denso), si $\overline{A}=X$.
        \item El espacio topológico $X$ es separable si tiene un subconjunto separable contable y denso. 
        \item Un punto de adherencia de $A\subseteq X$ es cualquier elemento de $\overline{A}$.
    \end{enumerate}
\end{definicion}

\begin{prop}
    Sea $A\subset B\implies A'\subset B'$.
    \begin{prop}
        Sea $A\subset B$ y sea $x\in A'\implies$ si $G$ es un abierto $\ni x\in G\implies (G-\{x\})\cap B\supset (G-\{x\})\cap A\neq \varnothing\implies x\in B'\implies A'\subset B'$.
    \end{prop}
\end{prop}

\begin{prop}
    Derivado de la unión $(A\cup B)'=A'\cup B' $
    \begin{dem}
        Por doble contención:
        \begin{itemize}
            \item $(\supseteq)$. A probar: $A'\cup B'\subset (A\cup B)'$. Sea $A\subseteq A\cup B$ y $B\subseteq A\cup B\implies A'\subseteq (A\cup B)'$ y $B'\subseteq (A\cup B)'\implies A'\cup B'\subseteq (A\cup B)'$
            \item $(\subseteq)$. A probar $(A\cup B)'\subseteq A'\cup B'\iff x\in (A\cup B)'\implies x\in A'\cup B'\iff$ si $x\not\in A'\cup B'\implies x\not\in (A\cup B)'$.
            \begin{itemize}
                \item Suponemos que $x\not\in A'\cup B'\implies x\not\in A'$ y $x\not\in B'\implies$ existen $G,H$ abiertos de $X\ni x\in G$ y $x\in H$ y $(G-\{x\})\cap A=\varnothing$ y $(H-\{x\})\cap B=\varnothing $ ya que $x\in G$ y $x\in H\implies x\in G\cap H$. Además, $G\cap H\subseteq G$ y $G\cap H\subseteq H$. Entonces $(G\cap H-\{x\})\cap A=\varnothing$ y $(G\cap H-\{x\})\cap B =\varnothing$. Por lo tanto, $(G\cap H-\{x\})\cap (A\cup B)=\varnothing$. 
            \end{itemize}
        \end{itemize}
    \end{dem}
\end{prop}
\begin{prop}
    $A\subseteq X$ es cerrado ssi $A'\subseteq A$.
    \begin{dem}
        Sea
        \begin{itemize}
            \item $(\implies)$
            \item $(\impliedby)$
        \end{itemize}
    \end{dem}
\end{prop}

\begin{prop}
    Sea $F$ un superconjunto cerrado de $A$, entonces $A'\subset F$. 
    \begin{dem}
        Como $A\subset F\implies A'\subset F'$. Como $F$ es cerrado, $F'\subset F\implies A'\subset F$.
    \end{dem}
\end{prop}

\begin{prop}
    $A\cup A'$ es cerrado. 
\end{prop}


\begin{problema}
    \begin{enumerate}
        \item 1. Demuestre que $f: X \rightarrow Y$ es continua ssi $f^{-1}\left(A^0\right) \subset\left[f^{-1}(A)\right]^0$, para cada $A \subset X$.
        \item  2. Considere a $\mathbb{R}$ con la topología usual. Pruebe que si cada función $f: X \rightarrow \mathbb{R}$ es continua, entonces $X$ es un espacio discreto.
        \item  3. Sea $f: X \rightarrow Y$ un mapeo entre espacios topológicos. Demuestre los enunciados siguientes:
        \begin{enumerate}
            \item 3.1. $f$ es cerrado ssi $\overline{f(A)} \subset f(\bar{A})$ para cada $A \subset X$.
            \item 3.2. $f$ es abierto ssi $f\left(A^0\right) \subset(f(A))^0$ para cada $A \subset X$.
        \end{enumerate}
        
        \item 4. Pruebe cada una de las siguientes es una propiedad topológica: Punto límite, Interior, Frontera, Vecindad
        \item Suponga que $X, Y$ son espacios topológicos tales que $X=\cup X_n \& Y=\cup Y_n$, donde $\left(X_n\right), \quad\left(Y_n\right)$ son sucesiones de conjuntos abiertos disjuntos en $X$ \& $Y$, respectivamente. Pruebe que, si $X_n$ es homemorfo a $Y_n$ para cada $n$, entonces $X$ \& $Y$ son homeomorfos.
    \end{enumerate}
\end{problema}

\begin{problema}
    1.
1.1. Si $A \subset X$, demuestre que la familia de todos los subconjuntos de $X$ que contienen a $A$, junto con el conjunto vacío $\emptyset$, es una topología sobre $X$.
1.2. ¿Qué topología resulta cuando $A=\emptyset$ ? ¿Y cuando $A=X$ ?
2.
2.1. Pruebe la operación $A \mapsto A^0$, en un espacio topológico $X$ tiene las propiedades siguientes:
a) $A^0 \subset A$
b) $\left(A^0\right)^0=A^0$
c) $A^0 \cap B^0=(A \cap B)^0$
d) $X^0=X$
e) $G$ es abierto ssi $G^0=G$
2.2. Conversamente, demuestre que si $X$ es un conjunto, cualquier mapeo de $\mathcal{P}(X)$ en $\mathcal{P}(X)$, tal que $A \mapsto A^0$ y que satisface de las propiedades a), b), c) y d), y si los conjuntos abiertos se definen en $X$ mediante e), entonces el resultado es una topología sobre $X$ en la cual el interior de un conjunto $A \subset X$ es $A^0$.
3. Pruebe que las topologías sobre un conjunto fijo $X$, parcialmente ordenadas por la inclusión, forman un retículo.
(Ayuda: Ver problema 3G de Willard).
4. Un subconjunto abierto $G$ en un espacio topológico es regularmente abierto ssi $G$ es el interior de su cerradura. Un subconjunto cerrado es regularmente cerrado ssi es la cerradura de su interior. Pruebe que:
4.1. El complemento de un conjunto regularmente abierto es regularmente cerrado y viceversa.
4.2. Si $A$ es un subconjunto cualquiera de un espacio topológico, entonces $(\bar{A})^0$ es regularmente abierto.
4.3. La intersección de dos conjuntos regularmente abiertos es regularmente abierto. ¿Se cumple esta propiedad en el caso de la unión?
5. Demuestre que si $\mathcal{B}$ es una base para una topología sobre $X$, entonces la topología generada por $\mathcal{B}$ es igual a la intersección de todas las topologías sobre $X$ que contienen a $B$.
\end{problema}









%----------------------------------------
\newpage

\end{document}