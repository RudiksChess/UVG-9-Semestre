\input{Configuraciones/paquetes}

%--------------------------

\begin{document}
 \thispagestyle{empty} 
    \begin{tabular}{p{15.5cm}}
    \begin{tabbing}
    \textbf{Universidad del Valle de Guatemala} \\\\
   \textbf{Estudiantes:} Rudik Rompich, Alejandro García Aguirre, Lisandro Toruño\\

    \end{tabbing}
    \begin{center}
        Teoría electromagnética 1 - Catedrático: Eduardo Álvarez\\
        \today
    \end{center}\\
    \hline
    \\
    \end{tabular} 
    \vspace*{0.3cm} 
    \begin{center} 
    {\Large \bf  Simulación
} 
        \vspace{2mm}
    \end{center}
    \vspace{0.4cm}
%--------------------------

\begin{problema}
1. Demuestre que $f: X \rightarrow Y$ es continua ssi $f^{-1}\left(A^0\right) \subset\left[f^{-1}(A)\right]^0$, para cada $A \subset X$.
2. Considere a $\mathbb{R}$ con la topología usual. Pruebe que si cada función $f: X \rightarrow \mathbb{R}$ es continua, entonces $X$ es un espacio discreto.
3. Sea $f: X \rightarrow Y$ un mapeo entre espacios topológicos. Demuestre los enunciados siguientes:
3.1. $f$ es cerrado ssi $\overline{f(A)} \subset f(\bar{A})$ para cada $A \subset X$.
3.2. $f$ es abierto ssi $f\left(A^0\right) \subset(f(A))^0$ para cada $A \subset X$.
4. Pruebe cada una de las siguientes es una propiedad topológica:
4.1. Punto limite
4.2. Interior
4.3. Frontera
4.4. Vecindad
5.
5.1. Investigue (No incluir en la solución)
5.1.1.La línea (o recta) de Sorgenfrey
5.1.2.El plano de Moore (o plano de Niemytzki)
5.2. Pruebe que la línea de Sorgenfrey y el plano de Niemytzki no son homeomorfos.
6. Suponga que $X, Y$ son espacios topológicos tales que $X=\cup X_n \& Y=\cup Y_n$, donde $\left(X_n\right), \quad\left(Y_n\right)$ son sucesiones de conjuntos abiertos disjuntos en $X$ \& $Y$, respectivamente. Pruebe que, si $X_n$ es homemorfo a $Y_n$ para cada $n$, entonces $X \& Y$ son homeomorfos.

%---------------------------
%\bibliographystyle{apa}
%\bibliography{referencias.bib}

\end{document}