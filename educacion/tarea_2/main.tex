\input{Configuraciones/paquetes}

%--------------------------

\begin{document}
 \thispagestyle{empty} 
    \begin{tabular}{p{15.5cm}}
    \begin{tabbing}
    \textbf{Universidad del Valle de Guatemala} \\\\
   \textbf{Estudiantes:} Rudik Rompich, Alejandro García Aguirre, Lisandro Toruño\\

    \end{tabbing}
    \begin{center}
        Teoría electromagnética 1 - Catedrático: Eduardo Álvarez\\
        \today
    \end{center}\\
    \hline
    \\
    \end{tabular} 
    \vspace*{0.3cm} 
    \begin{center} 
    {\Large \bf  Simulación
} 
        \vspace{2mm}
    \end{center}
    \vspace{0.4cm}
%--------------------------
\textbf{Instrucciones:}\bigbreak

\begin{enumerate}
    \item Elegir un modelo de cada tipo de espacios presentado: 
    \begin{itemize}
        \item Espacios físicos dentro de la institución
        \item Espacios físicos fuera de la institución
        \item Espacios virtuales
    \end{itemize}

    \item Describir cada modelo elegido  (mínimo 1 párrafo por cada espacio).
    \item Agregar, ¿Cómo sería la gestión de cada uno de estos espacios descritos?
    
\end{enumerate}

\section{Espacio físico: Laboratorio}
\subsection{Descripción}
Un laboratorio (científico, en mi caso) debería ser un espacio enfocado en la creatividad, en la innovación y en la experimentación. Debería estar distribuido en múltiples mesas de trabajo, pizarrones, áreas de uso común y áreas de uso especializado. El espacio debería ser amplio y bien distribuido. Además, es imperativo que hayan advertencias, reglas y precauciones bien señalizadas en las entradas del laboratorio, para evitar cualquier accidente o mal uso del mismo. 

\subsection{¿Cómo sería la gestión de cada uno de estos espacios descritos?}

Dependería mucho del tipo de materiales o mobiliario que se manejen en el laboratorio, en función de eso, la gestión podría ser manejado por un administrador (no necesariamente que esté presente) y una persona (o múltiples personas) que cuiden el laboratorio en horarios hábiles.


\section{Espacio fuera: Museo}
\subsection{Descripción}

Este espacio debería contener los aspectos más esenciales del tópico principal del museo. El museo debería contener visitas guiadas o por lo menos indicaciones o diseñas para hacer un recorrido del mismo. El museo debería contener proporcionalmente una división entre un aspecto clásico y un aspecto moderno, aunque definitivamente dependerá del tema del que trate. Además, un añadido podría ser cambiar paulatinamente los temas del museo, para promover el retorno al museo de los usuarios.


\subsection{¿Cómo sería la gestión de cada uno de estos espacios descritos?}

Dependerá del tamaño del museo, pero algunos aspectos importantes será tener una administración de la contabilidad, limpieza, personal de apoyo, etcétera. Para la gestión de un museo, se podría necesitar bastantes personas para suplir los diferentes aspectos. 

\section{Virtual: Realidad virtual}
\subsection{Descripción}
Este debería ser un espacio innovador y de última generación, algo vanguardista. En esencia, este espacio de aprendizaje debería aprovechar las últimas tecnologías en el campo de la virtualidad, lo que permitiría aprender desde cualquier lugar en el que se decida, tener una experiencia similar a la presencial y además un enfoque a ser autodidactas. 

\subsection{¿Cómo sería la gestión de cada uno de estos espacios descritos?}

De preferencia, debería consistir en automatización de todos los procesos administrativos. No se debería limitar las gestiones en horario hábil, la única dificultad debería estar enfocado en asistir únicamente a clases. 
%---------------------------
%\bibliographystyle{apa}
%\bibliography{referencias.bib}

\end{document}