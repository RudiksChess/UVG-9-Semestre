\documentclass[a4paper,12pt]{article}
\usepackage[top = 2.5cm, bottom = 2.5cm, left = 2.5cm, right = 2.5cm]{geometry}
\usepackage[T1]{fontenc}
\usepackage[utf8]{inputenc}
\usepackage{multirow} 
\usepackage{booktabs} 
\usepackage{graphicx}
\usepackage[spanish]{babel}
\usepackage{setspace}
\setlength{\parindent}{0in}
\usepackage{float}
\usepackage{fancyhdr}
\usepackage{amsmath}
\usepackage{amssymb}
\usepackage{amsthm}
\usepackage[numbers]{natbib}
\newcommand\Mycite[1]{%
	\citeauthor{#1}~[\citeyear{#1}]}
\usepackage{graphicx}
\usepackage{subcaption}
\usepackage{booktabs}
\usepackage{etoolbox}
\usepackage{minibox}
\usepackage{hyperref}
\usepackage{xcolor}
\usepackage[skins]{tcolorbox}
%---------------------------

\newtcolorbox{cajita}[1][]{
	 #1
}

\newenvironment{sol}
{\renewcommand\qedsymbol{$\square$}\begin{proof}[\textbf{Solución.}]}
	{\end{proof}}

\newenvironment{dem}
{\renewcommand\qedsymbol{$\blacksquare$}\begin{proof}[\textbf{Demostración.}]}
	{\end{proof}}

\newtheorem{problema}{Problema}
\newtheorem{definicion}{Definición}
\newtheorem{ejemplo}{Ejemplo}
\newtheorem{teorema}{Teorema}
\newtheorem{corolario}{Corolario}[teorema]
\newtheorem{lema}[teorema]{Lema}
\newtheorem{prop}{Proposición}
\newtheorem*{nota}{\textbf{NOTA}}
\renewcommand\qedsymbol{$\blacksquare$}
\usepackage{svg}
\usepackage{pgfplots}
\pgfplotsset{compat=1.11}

\usepackage{tikz}
\usetikzlibrary{calc}

\usetikzlibrary{patterns}
\usepackage[framemethod=default]{mdframed}
\global\mdfdefinestyle{exampledefault}{%
linecolor=lightgray,linewidth=1pt,%
leftmargin=1cm,rightmargin=1cm,
}




\newenvironment{noter}[1]{%
\mdfsetup{%
frametitle={\tikz\node[fill=white,rectangle,inner sep=0pt,outer sep=0pt]{#1};},
frametitleaboveskip=-0.5\ht\strutbox,
frametitlealignment=\raggedright
}%
\begin{mdframed}[style=exampledefault]
}{\end{mdframed}}
\newcommand{\linea}{\noindent\rule{\textwidth}{3pt}}
\newcommand{\linita}{\noindent\rule{\textwidth}{1pt}}

\AtBeginEnvironment{align}{\setcounter{equation}{0}}
\pagestyle{fancy}

\fancyhf{}









%----------------------------------------------------------
\lhead{\footnotesize Geometría diferencial}
\rhead{\footnotesize  Rudik Roberto Rompich}
\cfoot{\footnotesize \thepage}


%--------------------------

\begin{document}
 \thispagestyle{empty} 
    \begin{tabular}{p{15.5cm}}
    \begin{tabbing}
    \textbf{Universidad del Valle de Guatemala} \\
    Departamento de Matemática\\
    Licenciatura en Matemática Aplicada\\\\
   \textbf{Estudiante:} Rudik Roberto Rompich\\
   \textbf{Correo:}  \href{mailto:rom19857@uvg.edu.gt}{rom19857@uvg.edu.gt}\\
   \textbf{Carné:} 19857
    \end{tabbing}
    \begin{center}
        Geometría diferencial - Catedrático: Alan Reyes\\
        \today
    \end{center}\\
    \hline
    \\
    \end{tabular} 
    \vspace*{0.3cm} 
    \begin{center} 
    {\Large \bf  Tarea
} 
        \vspace{2mm}
    \end{center}
    \vspace{0.4cm}
%--------------------------

\textbf{Instrucciones:}
\begin{enumerate}
    \item Investiga sobre alguna metodología innovadora para la enseñanza-aprendizaje en espacios físicos. Para ello, busca uno o más artículos o papers académicos. Metodología elegida: \textbf{Aprendizaje basado en problemas}.
    
    \item Crea un documento Word
    
    \item Copia y completa el siguiente organizador gráfico, según tus hallazgos.
    \begin{itemize}
        \item ¿Qué dice el texto sobre la metodología?(Escribir datos utilizando
        paráfrasis)
        
        \begin{proof}[Respuesta]
            De acuerdo a \citeauthor{morales2004aprendizaje} (\citeyear{morales2004aprendizaje}) la metodología se resumen en: 
            \begin{itemize}
                \item Este fue un método desarrollado en la escuela de medicina de la universidad de McMaster. 
                \item El método se centra en el aprendizaje significativo a través de mejorar ciertas habilidades y competencias. 
                \item La esencia del método es trabajar en grupos pequeños para resolver un problema. 
                \item Este es un método centrado en el aprendizaje autónomo. 
            \end{itemize}
        \end{proof}
        
        \item ¿Qué elementos  debería incluirse en el  espacio de  aprendizaje, de  acuerdo a la metodología?	
        \begin{proof}[Respuesta]
            Los elementos que debe incluirse en el espacio de aprendizaje son los siguientes, considerando a  \citeauthor{poot2013retos} (\citeyear{poot2013retos}):
            \begin{itemize}
                \item El maestro o tutor no tiene un rol relevante, es el alumno quien guía su aprendizaje. 
                \item El método se basa en trabajar en equipo, siempre preservando que los grupos sean pequeños. 
                \item La discusión debe ser un elemento esencial al usar este método.
                \item El aprendizaje es activo, son los mismos estudiantes quienes buscan nuevo conocimiento. 
            \end{itemize}
        \end{proof}
        
        \item ¿Qué me gustaría  indagar a mayor  profundidad?
        \begin{proof}[Respuesta]
            Realmente, me gustaría indagar más en esta metodología y de alguna manera adquirir nuevas estrategias para mi propio aprendizaje. Esencialmente, sería bueno encontrar posibles utilidades en mi vida cotidiana. 
        \end{proof}
    \end{itemize}
    
    
    \item Contesta la pregunta: ¿Cómo se relaciona la intencionalidad del espacio de aprendizaje con las metodologías mencionadas en las fuentes de información? 
    \begin{proof}[Respuesta] 
        La intencionalidad del espacio de aprendizaje con el aprendizaje basado en problemas tiene un rol importante, ya que dicha metodología de aprendizaje, como dice \citeauthor{morales2004aprendizaje} (\citeyear{morales2004aprendizaje}) basa sus características en el aprendizaje autónomo en pequeños grupos. Por lo tanto, el espacio de aprendizaje, toma un rol importante en diversos aspectos, ya que se busca un espacio flexible y también un ambiente que promueva la discusión.

        Considerando la crítica de \citeauthor{morales2004aprendizaje} (\citeyear{morales2004aprendizaje}) a la metodología expositiva, en donde se menciona que es la metodología normal o usual en la mayoría de instituciones educativas en donde el mobiliario es estático y que la metodología se centra en contenidos y no interacciones entre alumnos y maestros; la metodología basado en problemas es todo lo contrario y el espacio de aprendizaje flexible se hace esencial, ya que se necesita cambiar frecuentemente el mobiliario de los salones de clase. Además, el maestro funciona como guía, entonces debe estar moviéndose frecuentemente y un correcto arreglo del mobiliario facilita esta tarea. 

        Por otra parte, el espacio de aprendizaje se hace bastante necesario considerando uno de los aspectos que menciona \citeauthor{poot2013retos} (\citeyear{poot2013retos}), la cual es: la discusión. Ya que para esto, es necesario mantener cierta estructuración del espacio de aprendizaje para hacer mucho más amena la experiencia de convivir y debatir ideas, principalmente mantener fluida la comunicación entre los miembros del pequeño grupo al que se es parte. 

        En resumen, el espacio de aprendizaje se relaciona con el aprendizaje basado en problemas en dos aspectos muy importantes: la flexibilidad y la apertura a la discusión. En estos aspectos se hace imperante contar con un espacio de aprendizaje óptima para que se logre desarrollar la metodología si ningún problema. 

    \end{proof}
\end{enumerate}

%(\citeauthor{brouwer1911beweis}, 
%\citeyear{brouwer1911beweis})

%---------------------------
\bibliographystyle{apalike}
\bibliography{referencias.bib}

\end{document}