\documentclass[a4paper,12pt]{article}
\usepackage[top = 2.5cm, bottom = 2.5cm, left = 2.5cm, right = 2.5cm]{geometry}
\usepackage[T1]{fontenc}
\usepackage[utf8]{inputenc}
\usepackage{multirow} 
\usepackage{booktabs} 
\usepackage{graphicx}
\usepackage[spanish]{babel}
\usepackage{setspace}
\setlength{\parindent}{0in}
\usepackage{float}
\usepackage{fancyhdr}
\usepackage{amsmath}
\usepackage{amssymb}
\usepackage{amsthm}
\usepackage[numbers]{natbib}
\newcommand\Mycite[1]{%
	\citeauthor{#1}~[\citeyear{#1}]}
\usepackage{graphicx}
\usepackage{subcaption}
\usepackage{booktabs}
\usepackage{etoolbox}
\usepackage{minibox}
\usepackage{hyperref}
\usepackage{xcolor}
\usepackage[skins]{tcolorbox}
%---------------------------

\newtcolorbox{cajita}[1][]{
	 #1
}

\newenvironment{sol}
{\renewcommand\qedsymbol{$\square$}\begin{proof}[\textbf{Solución.}]}
	{\end{proof}}

\newenvironment{dem}
{\renewcommand\qedsymbol{$\blacksquare$}\begin{proof}[\textbf{Demostración.}]}
	{\end{proof}}

\newtheorem{problema}{Problema}
\newtheorem{definicion}{Definición}
\newtheorem{ejemplo}{Ejemplo}
\newtheorem{teorema}{Teorema}
\newtheorem{corolario}{Corolario}[teorema]
\newtheorem{lema}[teorema]{Lema}
\newtheorem{prop}{Proposición}
\newtheorem*{nota}{\textbf{NOTA}}
\renewcommand\qedsymbol{$\blacksquare$}
\usepackage{svg}
\usepackage{pgfplots}
\pgfplotsset{compat=1.11}

\usepackage{tikz}
\usetikzlibrary{calc}

\usetikzlibrary{patterns}
\usepackage[framemethod=default]{mdframed}
\global\mdfdefinestyle{exampledefault}{%
linecolor=lightgray,linewidth=1pt,%
leftmargin=1cm,rightmargin=1cm,
}




\newenvironment{noter}[1]{%
\mdfsetup{%
frametitle={\tikz\node[fill=white,rectangle,inner sep=0pt,outer sep=0pt]{#1};},
frametitleaboveskip=-0.5\ht\strutbox,
frametitlealignment=\raggedright
}%
\begin{mdframed}[style=exampledefault]
}{\end{mdframed}}
\newcommand{\linea}{\noindent\rule{\textwidth}{3pt}}
\newcommand{\linita}{\noindent\rule{\textwidth}{1pt}}

\AtBeginEnvironment{align}{\setcounter{equation}{0}}
\pagestyle{fancy}

\fancyhf{}









%----------------------------------------------------------
\lhead{\footnotesize Geometría diferencial}
\rhead{\footnotesize  Rudik Roberto Rompich}
\cfoot{\footnotesize \thepage}


%--------------------------

\begin{document}
 \thispagestyle{empty} 
    \begin{tabular}{p{15.5cm}}
    \begin{tabbing}
    \textbf{Universidad del Valle de Guatemala} \\
    Departamento de Matemática\\
    Licenciatura en Matemática Aplicada\\\\
   \textbf{Estudiante:} Rudik Roberto Rompich\\
   \textbf{Correo:}  \href{mailto:rom19857@uvg.edu.gt}{rom19857@uvg.edu.gt}\\
   \textbf{Carné:} 19857
    \end{tabbing}
    \begin{center}
        Geometría diferencial - Catedrático: Alan Reyes\\
        \today
    \end{center}\\
    \hline
    \\
    \end{tabular} 
    \vspace*{0.3cm} 
    \begin{center} 
    {\Large \bf  Tarea
} 
        \vspace{2mm}
    \end{center}
    \vspace{0.4cm}
%--------------------------



\begin{problema}
    Tenemos: 
    \begin{enumerate}
        \item Pruebe que toda superficie regular compacta $S$ posee un punto elíptico.
        \begin{dem}
            A probar: Dada la superficie $S$, debemos encontrar un punto tal que $\det(dN_p)>0$. Considerando la demostración de Do Carmo, tenemos por hipótesis, $S$ es una superficie compacta regular $\implies$ Por la compacidad de $S\implies S$ es acotada. $\implies$ Existen esferas en $\mathbb{R}^3$, centradas en un punto fijo $O \in \mathbb{R}^3$, de tal manera que $S$ se encuentra en el interior de la región limitada por cualquiera de estas esferas. $\implies$ Consideremos el conjunto de todas estas esferas. Dejemos que $r$ sea el ínfimo de sus radios, y que $\Sigma$ sea una esfera de radio $r$ centrada en $O$.
            $\implies$ Por definición de ínfimo y dado que $S$ está contenida en todas las esferas de mayor radio, se sigue que $\Sigma$ y $S$ deben tener al menos un punto en común, digamos $p$. $\implies$ La construcción de $\Sigma$ implica que el plano tangente a $\Sigma$ en $p$ intersecta $S$ solo en el punto común $p$, en un entorno de $p$. Por lo tanto, $\Sigma$ y $S$ son tangentes en $p$. $\implies$ Al comparar secciones normales en $p$, se deduce que cualquier curvatura normal de $S$ en $p$ es mayor o igual a la curvatura correspondiente de $\Sigma$ en $p$.$\implies$ La definición de $\Sigma$ como una esfera de radio $r$ implica que $K_{\Sigma}(p) > 0$ para todos los $p$. Entonces tenemos, $K_S(p) \geq K_{\Sigma}(p)$ y $K_{\Sigma}(p) > 0\implies K_S(p) > 0$. $\implies$ Dado que $K_S(p) > 0$ en el punto $p$, por definición, $p$ es un punto elíptico de $S$.
        \end{dem}
        \item Muestre que toda superficie regular compacta $S$, con característica $\chi(S) 
        \leq 0$ posee un punto hiperbólico.
        \begin{dem}
            A probar: $S$ tiene un punto hiperbólico $\iff \det(dN_p)<0\iff K<0$. Por hipótesis, $S$ es una superficie regular compacto con $\chi(S)\leq 0\implies$ Por Gauss-Bonnet para superficies compactas,
            $$\int_S KdS=2\pi \chi(S)$$
            Pero en este caso, tenemos que $\chi(S)\leq 0$, lo que implica que 
            $$\int_S KdS\leq 0$$
            Pero este caso, solo se puede dar si $K<0$. Por lo tanto, $S$ tiene un punto hiperbólico. 
        \end{dem}
    \end{enumerate}
\end{problema}

\begin{problema}Compruebe que no existe superficie $\mathbf{x}(u, v)$ tal que $E=G=1, F=0$ y que $e=1, g=-1, f=0$.
    \begin{sol}
        Por reducción al absurdo, sabemos que: 
        $$K = \frac{eg - f^2}{EG - F^2}= \frac{1 \cdot (-1) - 0}{1 - 0} = -1$$
        Pero, en la tarea anterior habíamos demostrado que si $F=0$, entonces
            $$K=-\frac{1}{\sqrt{2EG}}\left(\left(\frac{E_v}{\sqrt{EG}}\right)_v+\left(\frac{G_u}{\sqrt{EG}}\right)_u\right)=-\frac{1}{\sqrt{2}}\left(0+0\right)=0$$
            Una contradicción. Por lo tanto, no existe dicha superficie. 
    \end{sol}
\end{problema}

\begin{problema} Justifique por qué las superficies siguientes no son localmente isométricas dos a dos:
    \begin{enumerate}
        \item la esfera $S^2$,
        
        \item el cilindro $S^1 \times \mathbb{R}$
        
        \item la silla $z=x^2-y^2$.
       
    \end{enumerate}
    \begin{sol}
        Para esto, hace falta verificar el teorema de Egrenium 
        \begin{cajita}
            \begin{teorema}[Egrenium]
                La curvatura gaussiana $K$ de una superficie es invariante por isometrías locales.
            \end{teorema}
        \end{cajita}
        Entonces, debemos calcular $K$ para cada una de las superficies dadas: 
        \begin{enumerate}
            \item la esfera $S^2$
            \begin{itemize}
                \item $\mathbf{x}(u,v)=(a\cos u \sin v,a\sin u\sin v, a\cos v)$
                \item $E=a\sin^2v, F=0,G=a^2$
                \item $e=a\sin^2v,f=0,g=a$
                \item $K=1/a^2$
            \end{itemize}
            
            \item el cilindro $S^1 \times \mathbb{R}$
            \begin{itemize}
                \item $\mathbf{x}(u,v)=(a\cos v, a\sin v,u)$
                \item $E=1, F=0,G=a^2$
                \item $e=0,f=0,g=a$
                \item $K=0$ 
            \end{itemize}
            \item la silla $z=x^2-y^2$.
            \begin{itemize}
                \item $\mathbf{x}(u,v)=(u,v,u^2-v^2)$
                \item $E=1+4u^2, F=-4uv,G=1+4v^2$
                \item $e=\frac{2}{\sqrt{4u^2+4v^2+1}},f=0,g=-\frac{2}{\sqrt{4u^2+4v^2+1}}$
                \item $K=\frac{-4}{(4u^2+4v^2+1)^2}$ 
            \end{itemize}
        \end{enumerate}
        Considerando que ninguna de las $K$ coincide, entonces por Egrenium, las superficies no son localmente isométricas dos a dos. 
    \end{sol}
\end{problema}
\begin{problema}
    Tenemos 
    \begin{enumerate}
        \item Dar la expresión para la ecuación de las geodésicas sobre el toro $\mathbb{T}^2$, con la parametrización usual
        $$
        \mathbf{x}(u, v)=((R+r \cos v) \cos u,(R+r \cos v) \sin u, r \sin v), \quad R>r>0, u, v \in(0,2 \pi) .
        $$
        \begin{sol}
            En el Do Carmo se hace la deducción de la ecuación de una geodésica para una superficie de revolución de la forma 
            $$x=f(v)\cos u\quad y=f(v)\sin u\quad z=g(v)$$
            En donde la ecuación de sus geodésicas es: 
            \begin{align*}
                \frac{du}{dv}&=\frac{1}{c}f\sqrt{\frac{f^2-c^2}{(f')^2+(g')^2}}
                \intertext{Reemplazando:}
                &=\frac{1}{c}(R+r\cos v)\sqrt{\frac{(R+r\cos v)^2-c^2}{(-r\sin v)^2+(r\cos v)^2}}\\
                &= \frac{1}{c}(R+r\cos v)\sqrt{\frac{(R+r\cos v)^2-c^2}{r^2}}\\
            \end{align*}
            
        \end{sol}
        \item Considere las curvas $\alpha(t)=\mathbf{x}(a, b t)$ y $\beta(t)=\mathbf{x}(a t, b)$, con $a, b \in \mathbb{R}, t \in \mathbb{R}$. Determinar para qué valores de $a, b$ estas curvas son geodésicas.
        \begin{sol}
            Tenemos \(\alpha(t) = \mathbf{x}(a, bt)\), la cual es una línea vertical si \(b \neq 0\), y la segunda, \(\beta(t) = \mathbf{x}(at, b)\), es una línea horizantal. Además, tenemos que 

            $$\frac{d^2 x^i}{dt^2} = 0$$

            Entonces, para cualquier valor de \(a\) y \(b\), las curvas \(\alpha(t)\) y \(\beta(t)\) son geodésicas.
        \end{sol}
    \end{enumerate}
    


Obs! No son las únicas geodésicas sobre el toro. El siguiente documento ilustra todas las familias de las geodésicas sobre $\mathbb{T}^2$. \href{http://ww.rdrop.com/ half/math/torus/torus.geodesics.pdf}{http://ww.rdrop.com/ half/math/torus/torus.geodesics.pdf}

\end{problema}

\begin{problema} Leer el punto 7, al final de la sección 4.5 del libro de Do Carmo (pp. 283-286). Entender el material, y probar el Teorema del Índice de Poincaré:
La suma de los índices de un campo vectorial diferenciable $X$ con puntos singulares sobre una superficie compacta $S$, es igual a la característica de Euler de $S$, esto es
$$
\sum_{\mathbf{p} \in S} I(X(\mathbf{p}))=\chi(S)
$$
\begin{dem}
    Esta demostración requiere una serie de pasos que se detallan en Do Carmo, en resumen, se utiliza lo siguiente: 

    \begin{enumerate}
        \item Se define el índice de un campo vectorial en un punto singular aislado, usando una parametrización ortogonal en ese punto y una curva positivamente orientada que lo rodea. Este índice es un número entero $I$, determinado a partir de la integral de la tasa de cambio del ángulo desde $\mathbf{x}_u$ hasta $v(t)$ alrededor de esta curva.

        \item Luego se demuestra que este índice $I$ es independiente de la elección de la parametrización, mostrando que es igual a la integral de la curvatura gaussiana $K$ sobre la región limitada por la curva, menos la mitad de cierto cambio de ángulo.
        
        \item Además, se demuestra que este índice $I$ no depende de la elección de la curva $\alpha$ que rodea el punto singular. Esto se hace considerando una deformación de una curva de este tipo en otra, y notando que el índice, al ser un número entero, no puede cambiar bajo una deformación continua.
        
        \item Luego señala que para puntos no singulares, el índice es cero.
        
        \item Para una superficie compacta orientada $S$ con un campo vectorial diferenciable $v$ que solo tiene puntos singulares aislados. Se observa que solo hay un número finito de estos puntos.
        
        \item Se introduce una triangulación de $S$ de tal manera que cada triángulo contiene a lo sumo un punto singular y su frontera no contiene ninguno. El teorema luego aplica una fórmula relacionando la integral de $K$ y la suma de los índices en cada triángulo.
        
        \item Al sumar estas fórmulas sobre todos los triángulos, y usando el teorema de Gauss-Bonnet, llega a la conclusión de que la suma de los índices de los puntos singulares es igual a la característica de Euler-Poincaré de $S$.
        $$
\sum_{\mathbf{p} \in S} I(X(\mathbf{p}))=\chi(S)
$$
    \end{enumerate}    
\end{dem}

\end{problema}
%---------------------------
%\bibliographystyle{apa}
%\bibliography{referencias.bib}

\end{document}