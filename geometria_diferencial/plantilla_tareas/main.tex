\documentclass[a4paper,12pt]{article}
\usepackage[top = 2.5cm, bottom = 2.5cm, left = 2.5cm, right = 2.5cm]{geometry}
\usepackage[T1]{fontenc}
\usepackage[utf8]{inputenc}
\usepackage{multirow} 
\usepackage{booktabs} 
\usepackage{graphicx}
\usepackage[spanish]{babel}
\usepackage{setspace}
\setlength{\parindent}{0in}
\usepackage{float}
\usepackage{fancyhdr}
\usepackage{amsmath}
\usepackage{amssymb}
\usepackage{amsthm}
\usepackage[numbers]{natbib}
\newcommand\Mycite[1]{%
	\citeauthor{#1}~[\citeyear{#1}]}
\usepackage{graphicx}
\usepackage{subcaption}
\usepackage{booktabs}
\usepackage{etoolbox}
\usepackage{minibox}
\usepackage{hyperref}
\usepackage{xcolor}
\usepackage[skins]{tcolorbox}
%---------------------------

\newtcolorbox{cajita}[1][]{
	 #1
}

\newenvironment{sol}
{\renewcommand\qedsymbol{$\square$}\begin{proof}[\textbf{Solución.}]}
	{\end{proof}}

\newenvironment{dem}
{\renewcommand\qedsymbol{$\blacksquare$}\begin{proof}[\textbf{Demostración.}]}
	{\end{proof}}

\newtheorem{problema}{Problema}
\newtheorem{definicion}{Definición}
\newtheorem{ejemplo}{Ejemplo}
\newtheorem{teorema}{Teorema}
\newtheorem{corolario}{Corolario}[teorema]
\newtheorem{lema}[teorema]{Lema}
\newtheorem{prop}{Proposición}
\newtheorem*{nota}{\textbf{NOTA}}
\renewcommand\qedsymbol{$\blacksquare$}
\usepackage{svg}
\usepackage{pgfplots}
\pgfplotsset{compat=1.11}

\usepackage{tikz}
\usetikzlibrary{calc}

\usetikzlibrary{patterns}
\usepackage[framemethod=default]{mdframed}
\global\mdfdefinestyle{exampledefault}{%
linecolor=lightgray,linewidth=1pt,%
leftmargin=1cm,rightmargin=1cm,
}




\newenvironment{noter}[1]{%
\mdfsetup{%
frametitle={\tikz\node[fill=white,rectangle,inner sep=0pt,outer sep=0pt]{#1};},
frametitleaboveskip=-0.5\ht\strutbox,
frametitlealignment=\raggedright
}%
\begin{mdframed}[style=exampledefault]
}{\end{mdframed}}
\newcommand{\linea}{\noindent\rule{\textwidth}{3pt}}
\newcommand{\linita}{\noindent\rule{\textwidth}{1pt}}

\AtBeginEnvironment{align}{\setcounter{equation}{0}}
\pagestyle{fancy}

\fancyhf{}









%----------------------------------------------------------
\lhead{\footnotesize Geometría diferencial}
\rhead{\footnotesize  Rudik Roberto Rompich}
\cfoot{\footnotesize \thepage}


%--------------------------

\begin{document}
 \thispagestyle{empty} 
    \begin{tabular}{p{15.5cm}}
    \begin{tabbing}
    \textbf{Universidad del Valle de Guatemala} \\
    Departamento de Matemática\\
    Licenciatura en Matemática Aplicada\\\\
   \textbf{Estudiante:} Rudik Roberto Rompich\\
   \textbf{Correo:}  \href{mailto:rom19857@uvg.edu.gt}{rom19857@uvg.edu.gt}\\
   \textbf{Carné:} 19857
    \end{tabbing}
    \begin{center}
        Geometría diferencial - Catedrático: Alan Reyes\\
        \today
    \end{center}\\
    \hline
    \\
    \end{tabular} 
    \vspace*{0.3cm} 
    \begin{center} 
    {\Large \bf  Tarea
} 
        \vspace{2mm}
    \end{center}
    \vspace{0.4cm}
%--------------------------


\begin{problema}
    1. a) Sea $\alpha(t)$ una curva parametrizada en $\mathbb{R}^{n}$, que no pasa por el origen $O$. Si $\alpha\left(t_{0}\right)$ es un punto del trazo de $\alpha$ que está más próximo a $O$, y $\alpha^{\prime}\left(t_{0}\right) \neq 0$ entonces $\alpha\left(t_{0}\right)$ es ortogonal a $\alpha^{\prime}\left(t_{0}\right)$

b) Sea $\alpha: I \rightarrow \mathbb{R}^{3}$ una curva parametrizada, con $\alpha^{\prime}(t) \neq 0, \forall t \in I$. Mostrar que $|\alpha(t)|$ es una constante $>0$ si, y sólo si, $\alpha(t)$ es ortogonal a $\alpha^{\prime}(t)$, para todo $t \in I$.
\end{problema}

\begin{problema}
    2. Considere la parametrización de la cicloide de radio $r$ vista en aula.

a) Calcular la longitud de arco de la cicloide en el primero de sus arcos, esto es correspondiente a una rotación completa del círculo.

b) Calcular el área bajo la curva (entre la curva y el eje $x$ ) para este arco de cicloide.
\end{problema}

\begin{problema}


3. Sea $\alpha:(0, \pi) \rightarrow \mathbb{R}^{2}$ la curva dada por

$$
\left(\sin t, \cos t+\log \tan \frac{t}{2}\right),
$$

donde $t$ es el ángulo que el eje $O y$ hace con el vector $\alpha^{\prime}(t)$. Esta curva se llama la tractriz (Figura en pág. 8 de Do Carmo). Mostrar que

- $\alpha$ es una curva parametrizada diferenciable, regular excepto en $t=\frac{\pi}{2}$.

- La longitud del segmento de la tangente a la tractriz, entre el punto de tangencia y el eje $O y$ es constante e igual a 1.

\end{problema}

\begin{problema}


4. Sea $\alpha$ una curva plana regular en coordenadas polares $(r, \varphi)$, dada por $r=r(\varphi)$. Usando la notación $r^{\prime}=\frac{\partial r}{\partial \varphi}$, verificar que la longitud de arco en el intervalo $\left[\varphi_{1}, \varphi_{2}\right]$ es

$$
s=\int_{\varphi_{1}}^{\varphi_{2}} \sqrt{r^{\prime 2}+r^{2}} d \varphi
$$

y que la curvatura está dada por

$$
\kappa(\varphi)=\frac{2 r^{\prime 2}-r r^{\prime \prime}+r^{2}}{\left(r^{\prime 2}+r^{2}\right)^{3 / 2}}
$$

\end{problema}

\begin{problema}

5. Calcular la curvatura de la espiral de Arquímedes, la cual está dada por $r(\varphi)=a \varphi, a$ constante (Figura $1(a))$.
\end{problema}

\begin{problema}

6. Para la espiral logarítmica, dada en coordenadas polares por por $r(t)=a e^{t}, \varphi(t)=b t, a, b$ constantes (Figura $\left.1(b)\right)$, probar lo siguiente:

a) La longitud de la curva en el intervalo $(-\infty, t]$ es proporcional al radio $r(t)$

b) $\alpha(t) \rightarrow 0$, cuando $t \rightarrow \infty$ y $\alpha$ tiene longitud de arco finita en el intervalo $\left[t_{0}, \infty\right)$.

c) El vector $\alpha(t)$ tiene ángulo constante con el vector tangente $\alpha^{\prime}(t)$.


\end{problema}

\begin{problema}
    7. Mostrar que la curva de menor longitud entre dos puntos $\mathbf{p}, \mathbf{q} \in \mathbb{R}^{n}$ es el segmento de recta que los une. (Sugerencia: ver las ideas en el Ejercicio 10, pág 11 de Do Carmo.) 


Figure 1: (a) espiral de Arquímedes, (b) espiral logarítmica.

\end{problema}

\begin{problema}


8. Probar que la curvatura y la torsión de una curva de Frenet $\alpha(t)$ en $\mathbb{R}^{3}$, parametrizada de forma arbitraria, están dadas por

$$
\kappa(t)=\frac{\left|\alpha^{\prime} \times \alpha^{\prime \prime}\right|}{\left|\alpha^{\prime}\right|^{3}}, \quad \tau(t)=\frac{\operatorname{det}\left(\alpha^{\prime}, \alpha^{\prime \prime}, \alpha^{\prime \prime \prime}\right)}{\left|\alpha^{\prime} \times \alpha^{\prime \prime}\right|^{2}} .
$$

En particular, en el caso de curvas planas,

$$
\kappa(t)=\frac{\operatorname{det}\left(\alpha^{\prime}, \alpha^{\prime \prime}\right)}{\left|\alpha^{\prime}\right|^{3}} .
$$

(Sugerencia: ver las ideas en el Ejercicio 12, pág 26 de Do Carmo.)
\end{problema}

\begin{problema}


9. Sea $\alpha$ la hélice en $\mathbb{R}^{3}$, dada por

$$
\alpha(t)=(a \cos t, a \sin t, b t), \quad a, b \in \mathbb{R}^{+} .
$$

Muestre que la curvatura y la torsión de $\alpha$ son constantes.
\end{problema}

\begin{problema}

10. Construir una curva plana, parametrizada por longitud de arco, cuya curvatura esté dada exactamente por $\kappa(s)=s^{-1 / 2}$.
\end{problema}

%---------------------------
%\bibliographystyle{apa}
%\bibliography{referencias.bib}

\end{document}