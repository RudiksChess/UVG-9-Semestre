\input{Configuraciones/paquetes}

%--------------------------

\begin{document}
 \thispagestyle{empty} 
    \begin{tabular}{p{15.5cm}}
    \begin{tabbing}
    \textbf{Universidad del Valle de Guatemala} \\\\
   \textbf{Estudiantes:} Rudik Rompich, Alejandro García Aguirre, Lisandro Toruño\\

    \end{tabbing}
    \begin{center}
        Teoría electromagnética 1 - Catedrático: Eduardo Álvarez\\
        \today
    \end{center}\\
    \hline
    \\
    \end{tabular} 
    \vspace*{0.3cm} 
    \begin{center} 
    {\Large \bf  Simulación
} 
        \vspace{2mm}
    \end{center}
    \vspace{0.4cm}
%--------------------------
\begin{problema}
     Describa un algoritmo con tiempo de ejecución $O\left(n \log _2 n\right)$ tal que, dados un conjunto $S$ de $n$ números enteros y un entero arbitrario $x$, determine si existen o no dos números en $S$ cuya suma sea exactamente $x$. Puede suponer que el arreglo está ordenado. 
     \begin{cajita}
        Hint: considere usar, como parte de su algoritmo, al algoritmo de búsqueda binaria, cuyo tiempo de ejecución es $O\left(\log _2 n\right)$.
     \end{cajita}
     \begin{sol}
        Sea $h(n)=n\log_2 n$. A encontrar: Un algoritmo con $O(h(n))=\{f(n): \exists c,n_0>0 | \forall n\geq n_0,0\leq f(n)\leq ch(n)\}$. Como pista, nos dicen que podemos integrar el algoritmo de búsqueda binaria. 
        \begin{cajita}
            El algoritmo de búsqueda binaria es un algoritmo que encuentra la posición de un número $T$ dentro de un arreglo ordenado $A$ con $n$ elementos, que está ordenado de forma ascendente. Se define como: 
        \begin{verbatim}
            binary_search(A, n, T):
                L = 0
                R = n - 1
                while L <= R: 
                    m = floor((L + R) / 2)
                    if A[m] < T:
                        L = m + 1
                    else if A[m] > T: 
                        R = m - 1
                    else:
                        return m
                return unsuccessful
        \end{verbatim}
        Este algoritmo tiene un tiempo de ejecución de  $O\left(\log _2 n\right)$. 
        \end{cajita}
        Entonces, esto nos indica que para obtener el tiempo de ejecución $h(n)$ debemos aplicarle un ciclo \textit{for} al algoritmo de búsqueda binaria que es $O\left(\log _2 n\right)$. Sin embargo, primero debemos encontrar un algoritmo que ponga en orden un arreglo $S$ con $n$ elementos que debe ir insertado en el algoritmo del binary-search. Nótese que en clase ya habíamos estudiado el algoritmo de Merge-Sort el cual tiene un tiempo de ejecución $O(h(n))$.
        \begin{cajita}
            El algoritmo de merge-sort es un algoritmo de ordenamiento, en donde $A$ es un array desornado, en donde $p$ es el primer elemento del array (se inicializa con $p=1$) y $r$ es la longitud del array: 
        \begin{verbatim}
        merge_sort(A,p,r):
            if p<r: 
                q = floor((p+r)/2)
                merge_sort(A,p,q)
                merge_sort(A,q+1,r)
                merge(A,p,q,r)
        \end{verbatim}
        Este algoritmo tiene un tiempo de ejecución de  $O\left(\log _2 n\right)$. 
        \end{cajita}

        En resumidas cuentas, si juntamos en un algoritmo el merge-sort con un ciclo \textit{for} del binary search, entonces obtendremos un algoritmo con tiempo de ejecución
        $$O(h(n))+O(h(n))=2O(h(n))=2O(n\log_2n)$$
        Pero el dos es una constante, entonces el algoritmo es de complejidad 
        $$O(n\log_2n)$$

        Entonces, el algoritmo propuesto, que dado un array $S$ de $n$ números enteros y un entero arbitrario $x$, que verifica que si existen o no dos números en $S$ cuya suma sea exactamente $x$. Es decir, se busca verificar que 
        $$x= S[i]+S[j],\quad i,j\in S$$
        Pero esto, esencialmente podría reescribirse como, al fijar un $S[j]$, debemos encontrar 
        $$S[j]=x-S[i]$$
        
        Entonces $S[j]$ es nuestra candidato para aplicarle el binary-search, ya que si lo encontramos, quiere decir que hay dos números en $S$ que al sumarlos dan $x$. 
        Entonces, definimos nuestro algoritmo de la siguiente manera: 
        
        \begin{verbatim}
            algoritmo_verificacion(S,x):
                p = 1
                n = length(S)
                S = merge_sort(S,p,n)
                for i=1 to n: 
                    S_j = x-S[i]
                    verificador_j = binary_search(S,r,S_j)
                    if verificajor_j != "unsuccessful": 
                        return "Verificacion exitosa"
                return "Verificacion fallida"
        \end{verbatim}

        El cual por la argumentación anterior, es $O(n\log_2 n)$.
     \end{sol}
\end{problema}

\begin{problema}La Regla de Horner dice que se puede evaluar un polinomio $P(x)=\sum_{k=0}^n a_k x^k$ de la siguiente manera:
$$
a_0+x\left(a_1+x\left(a_2+\cdots+x\left(a_{n-1}+x a_n\right) \ldots\right)\right)
$$
El siguiente trozo de pseudocódigo implementa esta regla para un conjunto de coeficientes $a_i$ dado:
\begin{verbatim}
    1. y=0
    2. for i=n downto 0:
    3.      y=a_i + x*y
\end{verbatim}
Calcule una cota ajustada para el tiempo de ejecución de este algoritmo.
\begin{sol}
    Analizamos por linea: 
    \begin{enumerate}
        \item $\Theta(1)$
        \item $\Theta(n)$
        \item $\Theta(1)$
    \end{enumerate}
    Usamos $\Theta$ ya que por definición Big-theta hace referencia a la cota ajustada. Entonces, la regla de Horner tiene una cota ajustada de $\Theta(n)$ ya que es el más grande del algoritmo. 
\end{sol}


\end{problema}

\begin{problema}
    Escriba código naïve para la evaluación de un polinomio (suponga que no hay una instrucción primitiva para calcular $x^y$ ). Compare las tasas de crecimiento de este código y el que implementa la Regla de Horner.
    \begin{cajita}
        Nota: una solución naïve se refiere a la implementación de una solución de la forma más simple posible, haciendo uso de funciones $u$ operaciones fundamentales y conocidas. Por ejemplo, la forma naïve de encontrar un elemento en un diccionario es ver la primera palabra y determinar si es la que buscamos. Si no es, pasamos a la siguiente y repetimos el proceso.
    \end{cajita} 
    \begin{sol}
        Analizamos primero los casos, para darnos una idea, considerando 
        $$P(x)=\sum_{k=0}^n a_k x^k$$
        \begin{enumerate}
            \item $n=0$, 
            \begin{align*}
                P(x)&=\sum_{k=0}^0 a_k x^k\\
                    &= 0
            \end{align*}
            \item $n=1$, 
            \begin{align*}
                P(x)&=\sum_{k=0}^1 a_k x^k\\
                &= a_0x^0 +a_1x^1 = a_0+a_1x^1
            \end{align*}
            \item $n=2$, 
            \begin{align*}
                P(x)&=\sum_{k=0}^2 a_k x^k\\
                &= a_0+a_1x^1 +a_2x^2\\
                &= a_0 +x(a_1+a_2x)
            \end{align*}
            \item $n=3$, 
            \begin{align*}
                P(x)&=\sum_{k=0}^3 a_k x^k\\
                &= a_0+a_1x^1 +a_2x^2 +a_3x^3\\
                &= a_0 +x(a_1+a_2x+a_3x^2)\\
                &= a_0 +x(a_1+x(a_2+a_3x))\\
            \end{align*}
        \end{enumerate}
        Entonces, tenemos el pseudocódigo:
        \begin{verbatim}
            DECLARE ARRAY a[n]
            DECLARE INTEGER N
            N = LENGTH(a)
            
            DECLARE INTEGER y
            y = 0
            
            FOR i = 0 TO N DO
                DECLARE INTEGER y_i
                y_i = 1
                
                FOR j = 1 TO i DO
                    y_i = y_i * x
                END FOR
                
                y = y + a[i] * y_i
            END FOR
        \end{verbatim}
    \end{sol}
\end{problema}

\begin{problema}
    Para dos funciones $f(n)$ y $g(n)$ demuestre que $$\max (f(n), g(n))=\Theta(f(n)+g(n)).$$
    \begin{dem}
        Por la definición de $\Theta$, 
        
        $$\Theta(f(n)+g(n))=\left\{h(n): \exists c_1,c_2,n_0>0|\forall n\geq n_0,  0\leq c_1\left(f(n)+g(n)\right)\leq h(n)\leq c_2\left(f(n)+g(n)\right) \right\},$$
        Es decir, lo que debemos probar es que $$h(n)=\max (f(n), g(n)),$$
        con sus respectivas constantes. Sin pérdida de generalidad, asúmase que $f(x)$ y $g(x)$ son asintóticamente positivas, es decir $f(x)\geq 0, n>n_1$  y $g(x)\geq 0, n>n_2$. Sea $n_0:= \max(n_1,n_2)$. Entonces, debemos encontrar las dos cotas, la superior e inferior: 
        \begin{itemize}
            \item Inferior. Por definición de $\max$, tenemos: 
            \begin{align*}
                f(x)&\leq \max(f(n),g(n))\\
                g(x)&\leq \max(f(n),g(n))\\
            \end{align*}
            Sumamos ambas expresiones: 
            $$f(x)+g(x)\leq \max(f(n),g(n))+\max(f(n),g(n))=2\max(f(n),g(n))$$
            Entonces: 
            $$ \frac{1}{2}\left(f(x)+g(x)\right)\leq \max(f(n),g(n))$$
            \item Superior, en este caso, es trivial: 
            $$(1)(f(x)+g(x))\geq \max(f(n),g(n))$$
        \end{itemize}
        Entonces, $c_1=\frac{1}{2}$, $c_2=1, n_0=\max(n_1,n_2)$, tal que: 
        $$\frac{1}{2}\left(f(x)+g(x)\right)\leq \max(f(n),g(n)) \leq f(x)+g(x)$$ 
        Por lo tanto, 
        $$\max (f(n), g(n))=\Theta(f(n)+g(n)).$$
    \end{dem}
\end{problema}

\begin{problema}
    Argumente por qué, para constantes reales cualesquiera $a$ y $b>0,(n+a)^b=\Theta\left(n^b\right)$.
    \begin{cajita}
        Hint: puede investigar o deducir la forma expandida $(n+a)^b$ para apoyar su respuesta.
    \end{cajita} 
    \begin{dem}
        Sea $\Theta\left(n^b\right)$, definido como:
        $$\Theta(n^b)=\left\{h(n): \exists c_1,c_2,n_0>0|\forall n\geq n_0,  0\leq c_1\left(n^b\right)\leq h(n)\leq c_2\left(n^b\right) \right\}$$
        Entonces debemos encontrar que $h(n)=(n+a)^b$ con sus respectivas constantes. Por hipótesis, $a,b\in \mathbb{R}$ y $b>0$. Entonces, analizamos la expresión $(n+a)^b$, en donde debemos encontrar su cota inferior y superior: 

        \begin{itemize}
            \item Superior. Nótese que $n\geq n_0> 0$, entonces por desigualdad triangular,  
            \begin{align*}
                n+a &\leq |n+a|\\
                &\leq \underbrace{|n|}_{n>0}+|a|\\
                &\leq n+|a|
                \intertext{Tenemos que, a partir de cierto punto $|a|\leq n $}
                &\leq n +n= 2n 
            \end{align*}
            Entonces, como $b>0$, elevamos a la $b$ ambos lados de la desigualdad: 
            $$(n+a)^b\leq (2n)^b=2^b n^b$$
            \item Inferior. Nótese que $n\geq n_0> 0$, entonces por desigualdad triangular,  
            \begin{align*}
                n+a &\geq |n-a|\\
                &\geq \underbrace{|n|}_{n>0}-|a|\\
                &\geq n-|a|
                \intertext{Tenemos que, a partir de cierto punto $|a|\leq n \implies 2|a|\leq 2n\implies 2|a|\leq n\leq 2n\implies 2|a|\leq n\implies |a|\leq n/2\implies -|a|\geq -n/2$}
                &\geq n-\frac{n}{2}=\frac{n}{2}
            \end{align*}
            Entonces, como $b>0$, elevamos a la $b$ ambos lados de la desigualdad: 
            $$(n+a)^b\geq \left(\frac{n}{2}\right)^b=\left(\frac{1}{2}\right)^bn^b$$
        \end{itemize}
        Por lo tanto, 
            \begin{align*}
                n_0=2|a|, \quad c_1=\left(\frac{1}{2}\right)^b,\quad c_2 = 2^b
            \end{align*}
            Tal que 
            $$\left(\frac{1}{2}\right)^b\left(n^b\right)\leq (n+a)^b\leq  2^b\left(n^b\right)$$
            Por lo tanto,
            $$(n+a)^b=\Theta\left(n^b\right)$$
    \end{dem}
\end{problema}

\begin{problema}
    ¿Es $2^{n+1}=O\left(2^n\right)$ ? ¿Es $2^{2 n}=O\left(2^n\right)$ ?
    \begin{sol}
        Por definición de $O(2^n)$, 
        $$O(2^n)=\left\{f(n):\exists n_0,c>0|\forall n\geq n_0, 0\leq f(n)\leq c 2^n\right\}$$
        Entonces, nótese que $2^{n+1}=2^n2$ y $2^{2n}=((2)^n)^2$. Es decir, $c=2$. Por lo tanto, 
        $$2^{n+1}=O\left(2^n\right)$$
        y
        $$2^{2 n}\neq O\left(2^n\right)$$
        ya que no se puede obtener la constante $c$. 
    \end{sol}
\end{problema}

\begin{problema} Demuestre las siguientes propiedades:
    \begin{enumerate}
        \item $f(n)=\Theta(g(n)) \Leftrightarrow f(n)=O(g(n)) \wedge f(n)=\Omega(g(n))$.
        \begin{dem}
            Por doble implicación: 
            \begin{itemize}
                \item $(\implies)$ Sea \begin{align*}
                    f(n)&=\Theta(g(n))\\
                        &= \left\{f(n):\exists c_1,c_2,n_0>0|\forall n\geq n_0,0\leq c_1g(n)\leq f(n)\leq c_2g(n)\right\}\\
                        &= \left\{f(n):\exists c_1,n_0>0|\forall n\geq n_0,0\leq c_1g(n)\leq f(n)\right\}\wedge\\
                        & \ \ \ \ \ \wedge \left\{f(n):\exists c_2,n_0>0|\forall n\geq n_0,0\leq f(n)\leq c_2g(n)\right\}\\
                        &= [f(n)=\Omega(g(n))]\wedge [f(n)=O(g(n))]\\
                        &= [f(n)=O(g(n))] \wedge [f(n)=\Omega(g(n))]
                \end{align*}
                \item $(\impliedby)$ Sea
                \begin{align*}
                    \left[f(n)=O(g(n)) \right]\wedge \left[f(n)=\Omega(g(n))\right] &= \left\{f(n):\exists c_2,n_1>0|\forall n\geq n_1,0\leq f(n)\leq c_2g(n)\right\} \wedge\\
                    &\ \ \ \  \wedge\left\{f(n):\exists c_1,n_2>0|\forall n\geq n_2,0\leq c_1g(n)\leq f(n)\right\}
                    \intertext{Sea $n_0=\max(n_1,n_2)$,}
                    &= \left\{f(n):\exists c_1,c_2,n_0>0|\forall n\geq n_0,\right.\\
                    &\ \ \ \ \ \left.0\leq c_1g(n)\leq f(n)\leq c_2g(n)\right\}\\
                    &= [f(n)=\Theta(g(n))]
                \end{align*}
                

            \end{itemize}
            Por lo tanto, 
            $$f(n)=\Theta(g(n)) \Leftrightarrow f(n)=O(g(n)) \wedge f(n)=\Omega(g(n))$$
        \end{dem}
        \item $o(g(n)) \cap \omega(g(n))=\varnothing$.
        \begin{dem}
            Sea 
            \begin{align*}
                f(n) &= o(g(n)) \cap \omega(g(n))\\
                  &= \left\{f(n):\exists c_2,n_1>0|\forall n\geq n_1,0\leq f(n)< c_2g(n)\right\} \cap\\
                 &\ \ \ \  \cap\left\{f(n):\exists c_1,n_2>0|\forall n\geq n_2,0\leq c_1g(n)< f(n)\right\}\\
                  \intertext{Sea $n_0=\max(n_1,n_2)$,}
                    &= \left\{f(n):\exists c_1,c_2,n_0>0|\forall n\geq n_0,0\leq c_1g(n)<f(n)< c_2g(n)\right\}\\
            \end{align*}
            Es decir que $o(g(n)) \cap \omega(g(n))$ nunca se llegan a intersectar a partir de un punto. Por lo tanto su interseccion es el $\varnothing$. 
        \end{dem}
        \item $f(n)=O(g(n)) \Rightarrow \log _2 f(n)=O\left(\log _2 g(n)\right)$, donde sepamos que $\log _2 g(n) \geq 1$ y  $f(n) \geq 1$ para $n$ suficientemente grande (i.e., para $n \geq n_0$ con algún $n_0$ ).
        \begin{sol}
            Sea 
            \begin{align*}
                f(n)&=O(g(n))\\
                &=  \left\{f(n):\exists c,n_0>0|\forall n\geq n_0,0\leq f(n)\leq cg(n)\right\}
            \end{align*}
            Como sabemos que $\log _2 g(n) \geq 1$ y  $f(n) \geq 1$ para $n\geq n_0$, es decir $$\left\{\log _2f(n):\exists c,n_0>0|\forall n\geq n_0,0\leq\log _2 f(n)\leq c\log _2g(n)\right\}$$
            Por lo tanto, 
            $$\log _2 f(n)=O\left(\log _2 g(n)\right).$$
        \end{sol}
    \end{enumerate}

\end{problema}






%---------------------------
%\bibliographystyle{apa}
%\bibliography{referencias.bib}

\end{document}