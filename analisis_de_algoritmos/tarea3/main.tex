\documentclass[a4paper,12pt]{article}
\usepackage[top = 2.5cm, bottom = 2.5cm, left = 2.5cm, right = 2.5cm]{geometry}
\usepackage[T1]{fontenc}
\usepackage[utf8]{inputenc}
\usepackage{multirow} 
\usepackage{booktabs} 
\usepackage{graphicx}
\usepackage[spanish]{babel}
\usepackage{setspace}
\setlength{\parindent}{0in}
\usepackage{float}
\usepackage{fancyhdr}
\usepackage{amsmath}
\usepackage{amssymb}
\usepackage{amsthm}
\usepackage[numbers]{natbib}
\newcommand\Mycite[1]{%
	\citeauthor{#1}~[\citeyear{#1}]}
\usepackage{graphicx}
\usepackage{subcaption}
\usepackage{booktabs}
\usepackage{etoolbox}
\usepackage{minibox}
\usepackage{hyperref}
\usepackage{xcolor}
\usepackage[skins]{tcolorbox}
%---------------------------

\newtcolorbox{cajita}[1][]{
	 #1
}

\newenvironment{sol}
{\renewcommand\qedsymbol{$\square$}\begin{proof}[\textbf{Solución.}]}
	{\end{proof}}

\newenvironment{dem}
{\renewcommand\qedsymbol{$\blacksquare$}\begin{proof}[\textbf{Demostración.}]}
	{\end{proof}}

\newtheorem{problema}{Problema}
\newtheorem{definicion}{Definición}
\newtheorem{ejemplo}{Ejemplo}
\newtheorem{teorema}{Teorema}
\newtheorem{corolario}{Corolario}[teorema]
\newtheorem{lema}[teorema]{Lema}
\newtheorem{prop}{Proposición}
\newtheorem*{nota}{\textbf{NOTA}}
\renewcommand\qedsymbol{$\blacksquare$}
\usepackage{svg}
\usepackage{pgfplots}
\pgfplotsset{compat=1.11}

\usepackage{tikz}
\usetikzlibrary{calc}

\usetikzlibrary{patterns}
\usepackage[framemethod=default]{mdframed}
\global\mdfdefinestyle{exampledefault}{%
linecolor=lightgray,linewidth=1pt,%
leftmargin=1cm,rightmargin=1cm,
}




\newenvironment{noter}[1]{%
\mdfsetup{%
frametitle={\tikz\node[fill=white,rectangle,inner sep=0pt,outer sep=0pt]{#1};},
frametitleaboveskip=-0.5\ht\strutbox,
frametitlealignment=\raggedright
}%
\begin{mdframed}[style=exampledefault]
}{\end{mdframed}}
\newcommand{\linea}{\noindent\rule{\textwidth}{3pt}}
\newcommand{\linita}{\noindent\rule{\textwidth}{1pt}}

\AtBeginEnvironment{align}{\setcounter{equation}{0}}
\pagestyle{fancy}

\fancyhf{}









%----------------------------------------------------------
\lhead{\footnotesize Geometría diferencial}
\rhead{\footnotesize  Rudik Roberto Rompich}
\cfoot{\footnotesize \thepage}


%--------------------------

\begin{document}
 \thispagestyle{empty} 
    \begin{tabular}{p{15.5cm}}
    \begin{tabbing}
    \textbf{Universidad del Valle de Guatemala} \\
    Departamento de Matemática\\
    Licenciatura en Matemática Aplicada\\\\
   \textbf{Estudiante:} Rudik Roberto Rompich\\
   \textbf{Correo:}  \href{mailto:rom19857@uvg.edu.gt}{rom19857@uvg.edu.gt}\\
   \textbf{Carné:} 19857
    \end{tabbing}
    \begin{center}
        Geometría diferencial - Catedrático: Alan Reyes\\
        \today
    \end{center}\\
    \hline
    \\
    \end{tabular} 
    \vspace*{0.3cm} 
    \begin{center} 
    {\Large \bf  Tarea
} 
        \vspace{2mm}
    \end{center}
    \vspace{0.4cm}
%--------------------------
\begin{problema}
    Considérese el \textit{Travelling Salesman Problem} (TSP):
\begin{verbatim}def tsp(G, start, node, visited):
    costs = []
    nextone = -1
    minactual = 10000000
    for edge in range(len(G[node])):
      if (edge not in visited):

        costs.append(G[node][edge]+tsp(G, start, edge, visited + [edge]))
  
        mintemporal = min(costs)
        if (minactual >= mintemporal):
          nextone = edge
          minactual = mintemporal
  
      if (len(costs)>0):
  
        visited.append(nextone)
  
        return min(costs)
        
      else:
  
        return G[node][start]
  
\end{verbatim}

Entonces, 
\begin{itemize}
    \item Divide-Conquer-Combine: 
    \begin{itemize}
        \item \textbf{Divide}: Es la parte en el que el problema se divide en subproblemas. En este caso tenemos al \textbf{for} que itera a cada \textbf{edge} y el condicional \begin{verbatim}
                if (edge not in visited)
        \end{verbatim}
        el cual subdivide el problema a los \textit{edges} no visitados. 
        \item \textbf{Conquer}: En esta parte es donde se ejecutan las llamadas recursivas. En este caso particular es cuando se ejecuta
        \begin{verbatim}
            tsp(G, start, edge, visited + [edge])
        \end{verbatim}
        \item \textbf{Combine}: Aquí es donde se juntas las llamadas recursivas, es decir, en este algoritmo es la parte en donde se agregan a la lista de costos: 
        \begin{verbatim}
costs.append(G[node][edge]+tsp(G, start, edge, visited + [edge]))
        \end{verbatim}
    \end{itemize}
    \item Relación de recurrencia de su tiempo de ejecución: 
    \begin{itemize}
        \item Aquí notamos que hay diversas parte del algoritmo que debemos tomar en cuenta,
        tenemos $n$ problemas (nodos) en el grafo completo, y cada problema (nodo) tiene $n-1$ subconjuntos posibles de problemas (nodos), esto podría expresarse como $T(n-1)$. Además, la recurrencia está dentro de un ciclo \textbf{for}. Las demás partes del algoritmo o son $\Theta(1)$ o $\Theta(n)$ podemos concluir que la recurrencia de este algoritmo es 
        $$T(n)=nT\left(n-1\right)+\Theta(n)$$
        Siendo más formalistas, tenemos 
        $$T(n)=\begin{cases}
            \Theta(1),& n=1\\
            T(n)=nT(n-1)+\Theta(n), & n>1
        \end{cases}$$

        Ahora, usando la técnica del árbol de recursión, tenemos: 
        \begin{figure}[H]
            \centering 
            \begin{tikzpicture}[level distance=1.5cm,
                level 1/.style={sibling distance=3cm},
                level 2/.style={sibling distance=1.5cm}]
                \node {T(n)}
                  child {node {T(n-1)}
                    child {node {T(n-2)}}
                    child {node {$\dots$}}
                    child {node {T(n-2)}}
                  }
                  child {node {$\dots$}}
                  child {node {T(n-1)}
                    child {node {T(n-2)}}
                    child {node {$\dots$}}
                    child {node {T(n-2)}}
                  };
              \end{tikzpicture}
        \end{figure}

        De esto, obtenemos: 
        \begin{align*}
            T(n) &= cn +cn(n-1)+cn(n-1)(n-2)+\cdots cn!
        \end{align*}
        Con lo que podemos concluir que el algoritmo tiene complejidad $$O(n!)$$
    \end{itemize}
    
\end{itemize}
\end{problema}

\begin{problema}
    Considere Merge Sort: 
    \begin{verbatim}
    Merge_Sort(A, p, r):
        if p < r:
            q= floor((p+r)/2)
            Merge-Sort(A,p,q)
            Merge-Sort(A,q+1,r)
            Merge(A,p,q,r)
    \end{verbatim}
    Entonces: 
    \begin{itemize}
        \item Divide-Conquer-Combine: 
        \begin{itemize}
            \item Divide: La parte donde se divide el algoritmo: 
            \begin{verbatim}
                        q= floor((p+r)/2)
                \end{verbatim}
            \item Conquer: Las llamadas recursivas, es decir: 
            \begin{verbatim}
                        Merge-Sort(A,p,q)
                        Merge-Sort(A,q+1,r)
                \end{verbatim}
            \item Combine: Donde se juntan las llamadas recursivas
            \begin{verbatim}
                        Merge(A,p,q,r)
                \end{verbatim}
        \end{itemize}
        \item Para este caso, la relación de recurrencia ya la habíamos encontrado en la clase, la cual es: 
        $$
T(n)= \begin{cases}\Theta(1) & \text { if } n=1 \\ 2 T(n / 2)+\Theta(n) & \text { if } n>1\end{cases}
$$
    \end{itemize}
    (O sea la respuesta que da el PDF del parcial, es correcta) y con Master Method, se había comprobado que la solución era $\Theta(n\log_2 n)$. Ahora bien, usando el método de substitución, tenemos: 
    \begin{itemize}
        \item Para $T(n)=O(n\log_2 n)$
        \begin{itemize}
            \item Paso inductivo, sea $T(n)\leq cn\log_2 n \implies T(n/2)\leq c\left(n/2\right)\log_2 (n/2)$. Por lo tanto, 
            \begin{align*}
                T(n)= 2 T(n / 2)+cn &\leq 2\left(c\left(n/2\right)\log_2 \left(n/2\right)\right)+cn \\
                &= cn\log_2\left(\frac{n}{2}\right)+cn \\
                &= cn(\log_2 n - \log_2 2)+cn\\
                &= cn\log_2n,\quad c\geq 2
            \end{align*}
            \item Paso base, a partir de $n\geq n_0=2$, $T(2)=4$ y la propiedad se cumple: 
            $$T(2)\leq c2\log_2(2)=2c,\quad c\geq 2$$ 
        \end{itemize}
        \item Para $T(n)=\Omega(n\log_2 n)$
        \begin{itemize}
            \item Paso inductivo, sea $T(n)\geq cn\log_2 n \implies T(n/2)\geq c\left(n/2\right)\log_2 (n/2)$. Por lo tanto, 
            \begin{align*}
                T(n)= 2 T(n / 2)+cn &\geq 2\left(c\left(n/2\right)\log_2 \left(n/2\right)\right)+cn \\
                &= cn\log_2\left(\frac{n}{2}\right)+cn \\
                &= cn(\log_2 n - \log_2 2)+cn\\
                &= cn\log_2n,\quad 0<c\leq 1
            \end{align*}
            \item Paso base, a partir de $n\geq n_0=2$, $T(2)=4$ y la propiedad se cumple: 
            $$T(2)\geq c2\log_2(2)=2c,\quad 0< c\leq 1$$ 
        \end{itemize}
    \end{itemize}
\end{problema}


%---------------------------
%\bibliographystyle{apa}
%\bibliography{referencias.bib}

\end{document}