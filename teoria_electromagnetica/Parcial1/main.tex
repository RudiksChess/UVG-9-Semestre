\input{Configuraciones/paquetes}

%--------------------------

\begin{document}
 \thispagestyle{empty} 
    \begin{tabular}{p{15.5cm}}
    \begin{tabbing}
    \textbf{Universidad del Valle de Guatemala} \\\\
   \textbf{Estudiantes:} Rudik Rompich, Alejandro García Aguirre, Lisandro Toruño\\

    \end{tabbing}
    \begin{center}
        Teoría electromagnética 1 - Catedrático: Eduardo Álvarez\\
        \today
    \end{center}\\
    \hline
    \\
    \end{tabular} 
    \vspace*{0.3cm} 
    \begin{center} 
    {\Large \bf  Simulación
} 
        \vspace{2mm}
    \end{center}
    \vspace{0.4cm}
%--------------------------

\begin{problema}
    Sean 
    \begin{align*}
        \vec{A} &=3 \vec{a}_x+1 \vec{a}_z = (3,0,1)\\
        \vec{B} &=-2 \vec{a}_x+5 \vec{a}_y-6 \vec{a}_z = (-2,5,-6)\\
        \vec{C} &=-4 \vec{a}_x-3 \vec{a}_y-2 \vec{a}_z = (-4,-3,-2)
    \end{align*}
\begin{enumerate}
    \item Encontrar el ángulo entre los vectores $\vec{B}$ y $\vec{C}$.
    \begin{sol}
        Sea
        \begin{align*}
            \implies & B\cdot C = |B||C|\cos \theta_{AB}\\
            \implies & \cos \theta_{AB} =\frac{ B\cdot C}{|B||C|}\\
            \implies & \theta_{AB} =\arccos\left(\frac{ B\cdot C}{|B||C|}\right)\\
        \end{align*}
        Para 
        \begin{align*}
            B\cdot C &= (-2,5,-6)\cdot (-4,-3,-2)\\
                     &= (-2)(-4)+(5)(-3)+(-6)(-2)\\
                     &= 8-15+12\\
                     &= 5
        \end{align*}
        y 
        \begin{align*}
            |B| &= |(-2,5,-6)| = \sqrt{4+25+36}= \sqrt{65}\\
            |C| &= |(-4,-3,-2)| = \sqrt{16+9+4}= \sqrt{29}
        \end{align*}
        Por lo tanto, 
        $$\theta_{AB} =\arccos\left(\frac{ B\cdot C}{|B||C|}\right)= \arccos\left(\frac{ 5}{\sqrt{65}\sqrt{29}}\right)=83.39^\circ $$
    \end{sol}
    \item Encontrar las componentes de $\vec{A}$ a lo largo de $\vec{B}$.
    \begin{sol}
        $$
\begin{aligned}
& A_B =A_B a_B=\left(A \cdot a_B\right) a_B= \\
& =\left((3,0,1) \cdot \frac{(-2,5,-6)}{\sqrt{(-2)^2+(5)^2+(-6)^2}}\right) \frac{(-2,5,-6)}{\sqrt{(-2)^2+(5)^2+(-6)^2}} \\
& =\frac{1}{65}\left((3)(-2)+(0)(5)+(1)(-6)\right)(-2,5,-6) \\
& =\frac{1}{65}(-6+0-6)(-2,5,-6) \\
& =\frac{-12}{65}(-2,5,-6) \\
&
\end{aligned}
$$
    \end{sol}
    \item Encontrar $\vec{B} \cdot(\vec{C} \times \vec{A})$.
    \begin{sol}
        Sea
        $$
\begin{aligned}
\Rightarrow C \times A= & \left|\begin{array}{ccc}
-4 & -3 & -2 \\
3 & 0 & 1
\end{array}\right| \\
= & {[(-3)(1)-(-2)(0)] a_x }- \\
& -[(-4)(1)-(-2)(3)] a_y+ \\
& +[(-4)(0)-(-3)(3)] a_z \\
& =[-3] a_x-[-4+6] a_y+[9] a_z \\
& =-3 a_x-2 a_y+9 a_z \\
& =(-3,-2,+9) \\
\Rightarrow B \cdot(C \times A) & =(-2,5,-6) \cdot(-3,-2,9) \\
& =(-2)(-3)+(5)(-2)+(-6)(9) \\
& =+6-10-54=+6-64=-58
\end{aligned}
$$
    \end{sol}
\end{enumerate}


\end{problema}

\begin{problema}
    Sea $\vec{A}=\rho^3\left(1-z^3\right) \vec{a}_\rho+\rho z^3 \cos (5 \phi) \vec{a}_\phi+7 \rho^2 z^4 \vec{a}_z$.
    
    Calcular:
    \begin{enumerate}
        \item El gradiente de $\vec{A}$.
        \begin{sol}
            No se puede aplicar, ya que no se cumple la definición de gradiente para un vector. 
        \end{sol}
        \item El rotor de $\vec{A}$.
        \begin{sol}
        
\begin{align*}
\nabla \times \vec{A} &=\frac{1}{\rho}\left|\begin{array}{ccc}
    \mathbf{a}_\rho & \rho \mathbf{a}_\phi & \mathbf{a}_z \\
    \frac{\partial}{\partial \rho} & \frac{\partial}{\partial \phi} & \frac{\partial}{\partial z} \\
    A_\rho & \rho A_\phi & A_z
    \end{array}\right|\\
&= {\left[\frac{1}{\rho} \frac{\partial A_z}{\partial \phi}-\frac{\partial A_\phi}{\partial z}\right] \mathbf{a}_\rho+\left[\frac{\partial A_\rho}{\partial z}-\frac{\partial A_z}{\partial \rho}\right] \mathbf{a}_\phi}+\frac{1}{\rho}\left[\frac{\partial\left(\rho A_\phi\right)}{\partial \rho}-\frac{\partial A_\rho}{\partial \phi}\right] \mathbf{a}_z\\
&= {\left[\frac{1}{\rho} \left(0\right)-\left(3\rho z^2 \cos (5 \phi)\right)\right] \mathbf{a}_\rho+
\left[-3z^2\rho^3-14\rho z^4\right] \mathbf{a}_\phi}
+\frac{1}{\rho}\left[2\rho z^3\cos(5\phi)-0\right] \mathbf{a}_z\\
&= {-\left(3\rho z^2 \cos (5 \phi)\right) \mathbf{a}_\rho-
\left[3z^2\rho^3+14\rho z^4\right] \mathbf{a}_\phi}
+\left[2 z^3\cos(5\phi)\right] \mathbf{a}_z\\
\end{align*}
        \end{sol}
    \end{enumerate}
\end{problema}

\begin{problema}
    Sea $\vec{D}=r^3 \cos (5 \phi) \vec{a}_r+28 \sen^4(4 \theta) \vec{a}_\theta+7 r^2 \vec{a}_{\phi}$
    \begin{enumerate}
        \item Calcular la divergencia de $\vec{D}$.
        \begin{sol}
            Sea
            \begin{align*}
                \nabla \cdot \mathbf{A} &=\frac{1}{r^2} \frac{\partial}{\partial r}\left(r^2 A_r\right)+\frac{1}{r \sin \theta} \frac{\partial}{\partial \theta}\left(A_\theta \sin \theta\right)+\frac{1}{r \sin \theta} \frac{\partial \mathrm{A}_\phi}{\partial \phi}\\
                &= \frac{1}{r^2} \frac{\partial}{\partial r}\left(r^5 \cos (5 \phi)\right)+\frac{1}{r \sin \theta} \frac{\partial}{\partial \theta}\left(28 \sen^4(4 \theta)\sin \theta\right)+\frac{1}{r \sin \theta} \left(0\right)\\
                &= \frac{1}{r^2} \left(5r^4 \cos (5 \phi)\right)+\frac{28}{r \sin \theta} \left[\sin\theta\left(16\sin^34\theta\cos4\theta\right)+\cos\theta \sin^4(4\theta )\right]
            \end{align*}
        \end{sol}
        \item Calcular el rotor de $\vec{D}$.
        \begin{sol}
            Sea
            \begin{align*}
                \nabla \times \mathbf{A} &=\frac{1}{r^2 \sin \theta}\left|\begin{array}{lll}
                    \mathbf{a}_r & r \mathbf{a}_\theta & r \sin \theta \mathbf{a}_\phi \\
                    \frac{\partial}{\partial r} & \frac{\partial}{\partial \theta} & \frac{\partial}{\partial \phi} \\
                    A_r & r A_\theta & r \sin \theta A_\phi
                    \end{array}\right|\\
                    &= \frac{1}{r \sin \theta}\left[\frac{\partial\left(A_\phi \sin \theta\right)}{\partial \theta}-\frac{\partial A_\theta}{\partial \phi}\right] \mathbf{a}_r +\\
                        & +\frac{1}{r}\left[\frac{1}{\sin \theta} \frac{\partial A_{\mathrm{r}}}{\partial \phi}-\frac{\partial\left(r A_\phi\right)}{\partial r}\right] \mathbf{a}_\theta+\frac{1}{r}\left[\frac{\partial\left(r A_\theta\right)}{\partial r}-\frac{\partial A_{\mathrm{r}}}{\partial \theta}\right] \mathbf{a}_\phi\\
                        &= \frac{1}{r \sin \theta}\left[\frac{\partial\left(7r^2 \sin \theta\right)}{\partial \theta}-0\right] \mathbf{a}_r +\\
                        & +\frac{1}{r}\left[\frac{1}{\sin \theta}\left(-5r^3\sin(5\phi)\right)-\frac{\partial\left(7r^3\right)}{\partial r}\right] \mathbf{a}_\theta+\frac{1}{r}\left[\frac{\partial\left(r 28 \sen^4(4 \theta)\right)}{\partial r}-0\right] \mathbf{a}_\phi\\
                        &= \frac{1}{r \sin \theta}\left[\left(7r^2 \cos \theta\right)\right] \mathbf{a}_r +\frac{1}{r}\left[\frac{\left(-5r^3\sin(5\phi)\right)}{\sin \theta}-21r^2\right] \mathbf{a}_\theta+\frac{1}{r}\left[\left(28 \sen^4(4 \theta)\right)\right] \mathbf{a}_\phi\\
                        &= \left[\left(7r \cot \theta\right)\right] \mathbf{a}_r +\left[\frac{\left(-5r^2\sin(5\phi)\right)}{\sin \theta}-21r\right] \mathbf{a}_\theta+\frac{1}{r}\left[\left(28 \sen^4(4 \theta)\right)\right] \mathbf{a}_\phi
            \end{align*}
        \end{sol}
    \end{enumerate}

\end{problema}

\begin{problema}
    Dado $\vec{T}=\left(\alpha x y-\beta z^3\right) \vec{a}_x+\left(3 x^2+\gamma z\right) \vec{a}_y+\left(3 x^2 z^2-y\right) \vec{a}_z$
Calcule:
\begin{enumerate}
    \item Si es irrotacional, encuentre $\alpha, \beta$ y $\gamma$.
    \begin{sol}
        Sea 
        $$\nabla\times \Vec{T}=0$$
        Entonces,
        \begin{align*}
            \nabla\times \Vec{T} &={\left[\frac{\partial T_z}{\partial y}-\frac{\partial \mathrm{T}_y}{\partial z}\right] \mathbf{a}_x+\left[\frac{\partial T_x}{\partial z}-\frac{\partial \mathrm{T}_z}{\partial x}\right] \mathbf{a}_y } +\left[\frac{\partial T_y}{\partial x}-\frac{\partial T_x}{\partial y}\right] \mathbf{a}_z\\
            &={\left[(-1)-\gamma\right] \mathbf{a}_x+\left[-3\beta z^2-6xz^2\right] \mathbf{a}_y } +\left[ 6x-\alpha x\right] \mathbf{a}_z\\
        \end{align*}
        Por lo tanto, 
        \begin{align*}
            \gamma &= -1\\
            \beta &= -2x\\
            \alpha &= 6
        \end{align*}
    \end{sol}
    \item Calcular la divergencia de $\vec{T}$ valuada en $(2,-1,-3)$
    \begin{sol}
        Tenemos: 
        $$\vec{T}=\left(6 x y+2xz^3\right) \vec{a}_x+\left(3 x^2- z\right) \vec{a}_y+\left(3 x^2 z^2-y\right) \vec{a}_z$$
        Y la divergencia: 
        \begin{align*}
            \nabla \cdot \mathbf{A} &=\frac{\partial A_x}{\partial x}+\frac{\partial A_y}{\partial y}+\frac{\partial A_z}{\partial z}\\
            &= (6y+2z^3)+(0)+(6x^2z)\\
            &= 6y+2z^3+6x^2z
        \end{align*}
        Evaluando: 
        $$\nabla \cdot \mathbf{A}(2,-1,-3)=6(-1)+2(-3)^3+6(2)^2(-3)=-132$$
    \end{sol}
\end{enumerate}

\end{problema}

\begin{problema}
    Sea $\vec{D}=2 x y \vec{a}_y+x^2 z \vec{a}_z$ y el paralelepípedo rectangular formado por $\mathrm{x}=0$ y $\mathrm{x}=1, \mathrm{y}=0$ y $\mathrm{y}=2, z=0$ y z=3. Mostrar si se cumple el teorema de la divergencia.
$$
\oint_S \vec{D} \cdot d \vec{S}=\int_{V o l} \nabla \cdot \vec{D} d V o l
$$
\begin{figure}[H]
    \centering 


\tikzset{every picture/.style={line width=0.75pt}} %set default line width to 0.75pt        

\begin{tikzpicture}[x=0.75pt,y=0.75pt,yscale=-1,xscale=1]
%uncomment if require: \path (0,300); %set diagram left start at 0, and has height of 300

%Shape: Axis 2D [id:dp5792005959832626] 
\draw  (224,170.2) -- (442,170.2)(245.8,11.8) -- (245.8,187.8) (435,165.2) -- (442,170.2) -- (435,175.2) (240.8,18.8) -- (245.8,11.8) -- (250.8,18.8) (278.8,165.2) -- (278.8,175.2)(311.8,165.2) -- (311.8,175.2)(344.8,165.2) -- (344.8,175.2)(377.8,165.2) -- (377.8,175.2)(410.8,165.2) -- (410.8,175.2)(240.8,137.2) -- (250.8,137.2)(240.8,104.2) -- (250.8,104.2)(240.8,71.2) -- (250.8,71.2)(240.8,38.2) -- (250.8,38.2) ;
\draw   ;
%Straight Lines [id:da5703162245053527] 
\draw    (245.8,170.2) -- (138.66,269.84) (224.36,195.6) -- (218.91,189.74)(200.19,218.08) -- (194.75,212.22)(176.03,240.55) -- (170.58,234.69)(151.87,263.02) -- (146.42,257.17) ;
\draw [shift={(137.2,271.2)}, rotate = 317.08] [color={rgb, 255:red, 0; green, 0; blue, 0 }  ][line width=0.75]    (10.93,-3.29) .. controls (6.95,-1.4) and (3.31,-0.3) .. (0,0) .. controls (3.31,0.3) and (6.95,1.4) .. (10.93,3.29)   ;
%Straight Lines [id:da39827761569943754] 
\draw [color={rgb, 255:red, 255; green, 0; blue, 0 }  ,draw opacity=1 ] [dash pattern={on 4.5pt off 4.5pt}]  (287.8,192.2) -- (311.8,170.2) ;
%Straight Lines [id:da9560701914257098] 
\draw [color={rgb, 255:red, 255; green, 0; blue, 0 }  ,draw opacity=1 ] [dash pattern={on 4.5pt off 4.5pt}]  (311.8,170.2) -- (312.2,71.4) ;
%Straight Lines [id:da7140415746333499] 
\draw [color={rgb, 255:red, 255; green, 0; blue, 0 }  ,draw opacity=1 ] [dash pattern={on 4.5pt off 4.5pt}]  (245.8,170.2) -- (246.2,71.4) ;
%Straight Lines [id:da5509378593102877] 
\draw [color={rgb, 255:red, 255; green, 0; blue, 0 }  ,draw opacity=1 ] [dash pattern={on 4.5pt off 4.5pt}]  (312.2,71.4) -- (246.2,71.4) ;
%Straight Lines [id:da3324514719237518] 
\draw [color={rgb, 255:red, 255; green, 0; blue, 0 }  ,draw opacity=1 ] [dash pattern={on 4.5pt off 4.5pt}]  (222.2,93.4) -- (246.2,71.4) ;
%Straight Lines [id:da7677821983284759] 
\draw [color={rgb, 255:red, 255; green, 0; blue, 0 }  ,draw opacity=1 ] [dash pattern={on 4.5pt off 4.5pt}]  (221.8,192.2) -- (222.2,93.4) ;
%Straight Lines [id:da45257869811143625] 
\draw [color={rgb, 255:red, 255; green, 0; blue, 0 }  ,draw opacity=1 ] [dash pattern={on 4.5pt off 4.5pt}]  (287.8,192.2) -- (221.8,192.2) ;
%Straight Lines [id:da10859873326539815] 
\draw [color={rgb, 255:red, 255; green, 0; blue, 0 }  ,draw opacity=1 ] [dash pattern={on 4.5pt off 4.5pt}]  (287.8,192.2) -- (288.2,93.4) ;
%Straight Lines [id:da45942647410772197] 
\draw [color={rgb, 255:red, 255; green, 0; blue, 0 }  ,draw opacity=1 ] [dash pattern={on 4.5pt off 4.5pt}]  (288.2,93.4) -- (312.2,71.4) ;
%Straight Lines [id:da055775836725740824] 
\draw [color={rgb, 255:red, 255; green, 0; blue, 0 }  ,draw opacity=1 ] [dash pattern={on 4.5pt off 4.5pt}]  (288.2,93.4) -- (230.2,93.4) -- (222.2,93.4) ;
%Straight Lines [id:da05608813009656577] 
\draw [color={rgb, 255:red, 255; green, 0; blue, 0 }  ,draw opacity=1 ] [dash pattern={on 4.5pt off 4.5pt}]  (221.8,192.2) -- (245.8,170.2) ;
%Straight Lines [id:da04068091753644265] 
\draw [color={rgb, 255:red, 255; green, 0; blue, 0 }  ,draw opacity=1 ] [dash pattern={on 4.5pt off 4.5pt}]  (311.8,170.2) -- (253.8,170.2) -- (245.8,170.2) ;

% Text Node
\draw (139.2,274.6) node [anchor=north west][inner sep=0.75pt]    {$x$};
% Text Node
\draw (442.2,170.4) node [anchor=north west][inner sep=0.75pt]    {$y$};
% Text Node
\draw (225.2,8.4) node [anchor=north west][inner sep=0.75pt]    {$z$};


\end{tikzpicture}
\end{figure}
\begin{sol}
    Sea 
    \begin{itemize}
        \item Sea
        \begin{align*} & \oint_s \vec{D} \cdot d \vec{S}=\oint_s\left(0,2 x y, x^2 z\right) \cdot d \vec{S} \\ 
            &= \left(\left.\int\right|_{y=0}+\left.\int\right|_{x=1}+\left.\int\right|_{y=2}+\left.\int\right|_{x=0}+\left.\int\right|_{z=3}+\left.\int\right|_{z=0}\right) \left(0,2 x y, x^2 z\right) \cdot d \vec{S}\\
             &= \left.\int\right|_{y=0}\left(0,2 x y, x^2 z\right) \cdot d \vec{S}+\left.\int\right|_{x=1}\left(0,2 x y, x^2 z\right) \cdot d \vec{S}\\&+\left.\int\right|_{y=2}\left(0,2 x y, x^2 z\right) \cdot d \vec{S}+\left.\int\right|_{x=0}\left(0,2 x y, x^2 z\right) \cdot d \vec{S}\\
             &+\left.\int\right|_{z=3}\left(0,2 x y, x^2 z\right) \cdot d \vec{S}+\left.\int\right|_{z=0}\left(0,2 x y, x^2 z\right) \cdot d \vec{S}\\
             &= \left.\int\right|_{y=0}\left(0,0, x^2 z\right) \cdot (0, d x d z,0)+\left.\int\right|_{x=1}\left(0,2y, z\right) \cdot (d y d z, 0, 0)\\&+\left.\int\right|_{y=2}\left(0,4 x, x^2 z\right) \cdot (0, d x d z, 0)+\left.\int\right|_{x=0}\left(0,0, 0\right) \cdot (d y d z, 0, 0)\\
             &+\left.\int\right|_{z=3}\left(0,2 x y, 3x^2 \right) \cdot(0,0, d x d y)+\left.\int\right|_{z=0}\left(0,2 x y, 0\right) \cdot (0, 0, d x d y)\\
             &= \int \int 4xdxdz + \int\int 3x^2 dxdy\\
             &= 4\int_0^1 xdx\int_0^3 dz+3\int_0^1 x^2 dx\int_0^2 dy\\
             &= (2)(3)+(1)(2)\\
             &=8
        \end{align*}
        \item Sea 
        \begin{align*}\int_v \nabla \cdot D & d v=\int_V\left(\frac{\partial A_x}{\partial x}+\frac{\partial A_y}{\partial y}+\frac{\partial A_z}{\partial z}\right)(d x d y d z) \\ & =\int_V\left(0+2 x+x^2\right) d x d y d z \\ & =\int_0^1\left(2 x+x^2\right) d x \int_0^2 d y \int_0^3 d z \\ & =\left.\left[\frac{2}{2} x^2+\frac{x^3}{3}\right]\right|_0 ^1(2)(3) \\ & =\left[(1)+\frac{1}{3}\right](2)(3)=\frac{4}{3}(2)(3)=8\end{align*}
    \end{itemize}
\end{sol}
\end{problema}
%---------------------------
%\bibliographystyle{apa}
%\bibliography{referencias.bib}

\end{document}