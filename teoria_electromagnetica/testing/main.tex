\documentclass[a4paper,12pt]{article}
\usepackage[top = 2.5cm, bottom = 2.5cm, left = 2.5cm, right = 2.5cm]{geometry}
\usepackage[T1]{fontenc}
\usepackage[utf8]{inputenc}
\usepackage{multirow} 
\usepackage{booktabs} 
\usepackage{graphicx}
\usepackage[spanish]{babel}
\usepackage{setspace}
\setlength{\parindent}{0in}
\usepackage{float}
\usepackage{fancyhdr}
\usepackage{amsmath}
\usepackage{amssymb}
\usepackage{amsthm}
\usepackage[numbers]{natbib}
\newcommand\Mycite[1]{%
	\citeauthor{#1}~[\citeyear{#1}]}
\usepackage{graphicx}
\usepackage{subcaption}
\usepackage{booktabs}
\usepackage{etoolbox}
\usepackage{minibox}
\usepackage{hyperref}
\usepackage{xcolor}
\usepackage[skins]{tcolorbox}
%---------------------------

\newtcolorbox{cajita}[1][]{
	 #1
}

\newenvironment{sol}
{\renewcommand\qedsymbol{$\square$}\begin{proof}[\textbf{Solución.}]}
	{\end{proof}}

\newenvironment{dem}
{\renewcommand\qedsymbol{$\blacksquare$}\begin{proof}[\textbf{Demostración.}]}
	{\end{proof}}

\newtheorem{problema}{Problema}
\newtheorem{definicion}{Definición}
\newtheorem{ejemplo}{Ejemplo}
\newtheorem{teorema}{Teorema}
\newtheorem{corolario}{Corolario}[teorema]
\newtheorem{lema}[teorema]{Lema}
\newtheorem{prop}{Proposición}
\newtheorem*{nota}{\textbf{NOTA}}
\renewcommand\qedsymbol{$\blacksquare$}
\usepackage{svg}
\usepackage{pgfplots}
\pgfplotsset{compat=1.11}

\usepackage{tikz}
\usetikzlibrary{calc}

\usetikzlibrary{patterns}
\usepackage[framemethod=default]{mdframed}
\global\mdfdefinestyle{exampledefault}{%
linecolor=lightgray,linewidth=1pt,%
leftmargin=1cm,rightmargin=1cm,
}




\newenvironment{noter}[1]{%
\mdfsetup{%
frametitle={\tikz\node[fill=white,rectangle,inner sep=0pt,outer sep=0pt]{#1};},
frametitleaboveskip=-0.5\ht\strutbox,
frametitlealignment=\raggedright
}%
\begin{mdframed}[style=exampledefault]
}{\end{mdframed}}
\newcommand{\linea}{\noindent\rule{\textwidth}{3pt}}
\newcommand{\linita}{\noindent\rule{\textwidth}{1pt}}

\AtBeginEnvironment{align}{\setcounter{equation}{0}}
\pagestyle{fancy}

\fancyhf{}









%----------------------------------------------------------
\lhead{\footnotesize Geometría diferencial}
\rhead{\footnotesize  Rudik Roberto Rompich}
\cfoot{\footnotesize \thepage}


%--------------------------

\begin{document}
 \thispagestyle{empty} 
    \begin{tabular}{p{15.5cm}}
    \begin{tabbing}
    \textbf{Universidad del Valle de Guatemala} \\
    Departamento de Matemática\\
    Licenciatura en Matemática Aplicada\\\\
   \textbf{Estudiante:} Rudik Roberto Rompich\\
   \textbf{Correo:}  \href{mailto:rom19857@uvg.edu.gt}{rom19857@uvg.edu.gt}\\
   \textbf{Carné:} 19857
    \end{tabbing}
    \begin{center}
        Geometría diferencial - Catedrático: Alan Reyes\\
        \today
    \end{center}\\
    \hline
    \\
    \end{tabular} 
    \vspace*{0.3cm} 
    \begin{center} 
    {\Large \bf  Tarea
} 
        \vspace{2mm}
    \end{center}
    \vspace{0.4cm}
%--------------------------


Sure! Let’s start by finding the electric field $\mathbf{E}$ for $r \geq a$. Since the charge distribution is spherically symmetric, we can use Gauss’s law to find the electric field. Gauss’s law states that the electric flux through any closed surface is equal to the total charge enclosed within the surface divided by the permittivity of free space. Mathematically, this can be written as:

$$ \oint_S \mathbf{E} \cdot d\mathbf{A} = \frac{Q_{enc}}{\epsilon_0} $$

where $S$ is any closed surface enclosing a charge $Q_{enc}$ and $\epsilon_0$ is the permittivity of free space.

In our case, we can choose a spherical Gaussian surface with radius $r>a$. The total charge enclosed within this surface is given by:

$$ Q_{enc} = \int_V \rho_\nu dV = \int_0^a 4\pi r2\rho_0\left(a2-r^2\right) dr = 4\pi\rho_0\int_0^a r2a2 - r^4 dr $$

Evaluating this integral gives us:

$$ Q_{enc} = 4\pi\rho_0\left[\frac{a4r}{3}-\frac{r5}{5}\right]_0^a = 4\pi\rho_0\left(\frac{2}{15}\right)a^5 $$

Since the electric field has spherical symmetry and points radially outward from the center of the sphere, we can write $\mathbf{E}=E_r®\hat{\mathbf{r}}$, where $\hat{\mathbf{r}}$ is a unit vector pointing radially outward from the center of the sphere. The electric flux through our Gaussian surface is then given by:

$$ \oint_S \mathbf{E} \cdot d\mathbf{A} = E_r® \oint_S dA = E_r® 4\pi r^2 $$

Substituting this into Gauss’s law and solving for $E_r®$ gives us:

Sure! So we have found that the electric field for $r \geq a$ is given by:

$$ \mathbf{E} = E_r®\hat{\mathbf{r}} = \frac{\rho_0 a^5}{3\epsilon_0 r^2}\hat{\mathbf{r}} $$

Now let’s find the electric potential $V$ for $r \geq a$. The electric potential at a point is defined as the work done per unit charge in bringing a test charge from infinity to that point. Mathematically, this can be written as:

$$ V® = -\int_\infty^r \mathbf{E} \cdot d\mathbf{l} $$

where $\mathbf{l}$ is the displacement vector along the path of integration.

In our case, we can choose a path of integration along the radial direction from infinity to $r$. Substituting our expression for $\mathbf{E}$ into this equation and evaluating the integral gives us:

$$ V® = -\int_\infty^r E_r(r’) dr’ = -\int_\infty^r \frac{\rho_0 a^5}{3\epsilon_0 r’^2} dr’ = -\left[-\frac{\rho_0 a^5}{3\epsilon_0 r’}\right]_\infty^r = \frac{\rho_0 a^5}{3\epsilon_0 r} $$

So the electric potential for $r \geq a$ is given by:

$$ V® = \frac{\rho_0 a^5}{3\epsilon_0 r} $$

------------------------------------------------


You are correct. I apologize for the mistake in my previous message. The total charge enclosed within a spherical Gaussian surface with radius $r<a$ is given by:

$$ Q_{enc} = \int_V \rho_\nu dV = \int_0^r 4\pi r2\rho_0\left(a2-r^2\right) dr = 4\pi\rho_0\int_0^r r2a2 - r^4 dr $$

Evaluating this integral gives us:

$$ Q_{enc} = 4\pi\rho_0\left[\frac{a2r3}{3}-\frac{r5}{5}\right]_0r = 4\pi\rho_0\left(\frac{2}{15}\right)a2r3 $$

Substituting this into Gauss’s law and solving for $E_r®$ gives us:

$$ E_r® = \frac{1}{4\pi r^2}\frac{Q_{enc}}{\epsilon_0} = \frac{\rho_0 a^2 r}{3 \epsilon_0} $$

So the electric field for $r \leq a$ is given by:

$$ \mathbf{E} = E_r®\hat{\mathbf{r}} = \frac{\rho_0 a^2 r}{3 \epsilon_0}\hat{\mathbf{r}} $$

----------------------------------


$$ Q_{enc} = \int_V \rho_\nu dV = \int_0^r 4\pi r2\rho_0\left(a2-r^2\right) dr = 4\pi\rho_0\int_0^r r2a2 - r^4 dr $$

Evaluating this integral gives us:

$$ Q_{enc} = 4\pi\rho_0\left[\frac{a2r3}{3}-\frac{r5}{5}\right]_0r = 4\pi\rho_0\left(\frac{2}{15}\right)a2r3 $$

Substituting this into Gauss’s law and solving for $E_r®$ gives us:

$$ E_r® = \frac{1}{4\pi r^2}\frac{Q_{enc}}{\epsilon_0} = \frac{\rho_0 a^2 r}{3 \epsilon_0} $$

So the electric field for $r \leq a$ is given by:

$$ \mathbf{E} = E_r®\hat{\mathbf{r}} = \frac{\rho_0 a^2 r}{3 \epsilon_0}\hat{\mathbf{r}} $$

------------------------------------


Para aplicar la fórmula que te di anteriormente en tu caso específico, primero debemos expresar los elementos de carga $dq_1$ y $dq_2$ en términos de la densidad de carga volumétrica $\rho_{vol}$ y el volumen $dV$. Podemos hacer esto utilizando la relación $dq = \rho_{vol} dV$. Luego, podemos reescribir la fórmula como:

$$ W = \frac{1}{2} \int \frac{k \rho_{vol}(r_1) \rho_{vol}(r_2) dV_1 dV_2}{|r_1 - r_2|} $$

donde $r_1$ y $r_2$ son las posiciones de los elementos de volumen $dV_1$ y $dV_2$, respectivamente.

Para evaluar esta integral, podemos utilizar coordenadas esféricas y expresar el elemento de volumen como $dV = r^2 sin(\theta) dr d\theta d\phi$


Si utilizamos coordenadas esféricas y expresamos el elemento de volumen como $dV = r^2 sin(\theta) dr d\theta d\phi$, podemos reescribir la integral para la energía necesaria para ensamblar la distribución de carga como:

$$ W = \frac{1}{2} \int \int \frac{k \rho_{vol}(r_1) \rho_{vol}(r_2) r_1^2 r_2^2 sin(\theta_1) sin(\theta_2) dr_1 dr_2 d\theta_1 d\theta_2 d\phi_1 d\phi_2}{|r_1 - r_2|} $$

Para evaluar esta integral, podemos dividirla en dos partes: una para $r<a$ y otra para $r>a$. Para $r<a$, podemos utilizar la expresión que nos has dado para $\rho_{vol}$ y sustituirla en la integral. Para $r>a$, $\rho_{vol} = 0$ y por lo tanto esa parte de la integral se anula.
%---------------------------
%\bibliographystyle{apa}
%\bibliography{referencias.bib}

\end{document}